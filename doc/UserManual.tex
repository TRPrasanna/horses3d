\documentclass[a4paper,10pt]{report}
%\usepackage[spanish]{babel}
\usepackage[utf8]{inputenc}
\usepackage{graphicx}
\usepackage{float}
\usepackage[section]{placeins}
\usepackage{multirow}
\usepackage{vmargin}
\usepackage{longtable}
\usepackage{amsmath}
\usepackage{amsfonts}
\usepackage{subfigure}
\usepackage{afterpage}
\usepackage[calcwidth]{titlesec}
\usepackage{verbatim}
\usepackage[hidelinks]{hyperref}
\usepackage{multicol}
\usepackage{pfnote}
\usepackage{fnpos}
\usepackage{color}
\usepackage{xcolor}
\usepackage{listings}
\usepackage{algorithm,algpseudocode}
\usepackage{epstopdf}
\usepackage{tcolorbox}
\usepackage{amssymb}
\usepackage{pgfplots}
\setmarginsrb{2cm}{2cm}{2cm}{2cm}{0cm}{0cm}{0cm}{0.5cm}%{left}{top}{right}{bottom}{headhgt}{}
%\numberwithin{equation}{section}
%Bibliography style:
\bibliographystyle{elsarticle-num}
%\biboptions{sort&compress}
%opening

\definecolor{mygreen}{RGB}{28,172,0} % color values Red, Green, Blue
\definecolor{mylilas}{RGB}{170,55,241}

%% Some math definitions:
\newcommand\norm[1]{\left\lVert#1\right\rVert}
\def\Res{\pmb{\mathfrak{R}}}


\title{\textbf{HORSES3D} \\ A \textbf{H}igh-\textbf{Or}der (DG) \textbf{S}pectral \textbf{E}lement \textbf{S}olver \\ \textbf{User Manual}}
\author{Numath group}

\begin{document}

\lstset{language=C++}

\maketitle

\tableofcontents

\chapter{Compiling the code} \label{sect:compiling}

\begin{itemize}
\item Clone the git repository or copy the source code into a desired folder.

\item Go to the folder Solver.

\item Run configure script.
\begin{lstlisting}[language=bash]
	$ ./configure
\end{lstlisting}
\item Install using the Makefile:
\begin{lstlisting}[language=bash]
	$ make all <<Options>>
\end{lstlisting}
with the desired options (bold are default):

\begin{itemize}
\item MODE=DEBUG/HPC/\textbf{RELEASE}
\item COMPILER=ifort/\textbf{gfortran}
\item COMM=PARALLEL/\textbf{SEQUENTIAL}
\item ENABLE\_THREADS=NO/\textbf{YES}
\item WITH\_PETSC=YES/\textbf{NO}
\item WITH\_METIS=YES/\textbf{NO}
\item WITH\_HDF5=YES/\textbf{NO}
\item WITH\_MKL=YES/\textbf{NO}
\end{itemize}

For example:
\begin{lstlisting}[language=bash]
	$ make all COMPILER=ifort COMM=PARALLEL
\end{lstlisting}

\item MODE=DEBUG/HPC/\textbf{RELEASE} flag enables various compiler flags for different levels of code optimization. Furthermore, MODE=HPC disables residual file writing to improve performance. 

\item The ENABLE\_THREADS=YES flag enables shared memory simulations using OpenMP.

\item The COM=PARALLEL flag enables distributed memory simulations using MPI.

\item The WITH\_METIS=YES flag activates METIS linking. To compile the code linking it with METIS (that is an option for creating the mesh partitions of MPI runs), it is needed that before compilation and running, an environment variable called METIS\_DIR is found. This variable must contain the path to the METIS installation folder (it must have been compiled with the same compiler as HORSES3D).

\item The WITH\_HDF5=YES flag activates HDF5 linking. To compile the code linking it with HDF5 (necessary for reading HOPR meshes), it is needed that before compilation and running, an environment variable called HDF5\_DIR is found. This variable must contain the path to the HDF5 installation folder (it must have been compiled with the same compiler as HORSES3D). In addition, the lib folder must be added to the environment variable LD\_LIBRARY\_PATH.

\item The WITH\_PETSC=YES flag activates PETSC linking. To compile the code linking it with PETSC, it is needed that before compilation and running, an environment variable called PETSC\_DIR is found. This variable must contain the path to the PETSC installation folder.

\item If you use \textit{environment modules}, it is advised to use the HORSES3D module file:\\
\begin{lstlisting}[language=bash]
	$ export MODULEPATH=$HORSES_DIR/utils/modulefile:$MODULEPATH
\end{lstlisting}
where \$HORSES\_DIR is the installation directory.

\item It is advised to run the \emph{make clean} or \emph{make allclean} command if some options of the compilation rutine needs to be changed and it has been compiled before.

\end{itemize}

\chapter{Input and Output Files}

DONT USE TABS!

\section{Input Files}
\begin{itemize}
\item Control file (*.control)
\item Mesh file (*.mesh / *.h5 / *.msh)
\item Polynomial order file (*.omesh)
\item Problem File (ProblemFile.f90)
\end{itemize}

{ \small
Notes on the GMSH format (*.msh) and general workflow using GMSH.
\begin{itemize}
\item Curved geometry supported up to polynomial order 5.
\item Curved geometry should be generated using following options: tools -> options -> mesh -> general -> element order.
\item HORSES3D can read mesh format 4.1 and 2.2 (legacy format).
\item The solution to most of the problems mesh reading is to load it in GMSH and export to format 2.2 to have a clean ASCII file.
\end{itemize}
}

\section{Output Files}
\begin{itemize}
\item Solution file (*.hsol)
\item Horses mesh file (*.hmesh)
\item Boundary information (*.bmesh)
\item Partition file (*.pmesh)
\item Polynomial order file (*.omesh)
\item Monitor files (*.volume / *.surface / *.residuals)
\end{itemize}

\chapter{Running a Simulation}

\section{Control File (*.control) - Overview}
The control file is the main file for running a simulation. A list of all the mandatory keywords for running a simulation and some basic optional keywords is presented in Table \ref{tab:runningkey}. The specific keywords are listed in the other chapters.

\begin{longtable}{|p{4cm}|p{10cm}|p{2.2cm}|}
\caption{General keywords for running a case.} \label{tab:runningkey} \\
\hline
\multicolumn{1}{|c|}{\textbf{Keyword}} & \multicolumn{1}{c|}{\textbf{Description}} & \multicolumn{1}{c|}{\textbf{Default value}} \\ \hline
\endfirsthead

\caption{General keywords for running a case - continued.} \\
\hline
\multicolumn{1}{|c|}{\textbf{Keyword}} & \multicolumn{1}{c|}{\textbf{Description}} & \multicolumn{1}{c|}{\textbf{Default value}} \\ \hline
\endhead

solution file name   & \textit{CHARACTER}: Path and name of the output file. The name of this file is used for naming other output files. & \textbf{Mandatory\ keyword} \\ \hline

simulation type        & \textit{CHARACTER}: Specifies if HORSES3D must perform a 'steady-state' or a 'time-accurate' simulation. &  'steady-state'\\ \hline

time integration & \textit{CHARACTER}: Can be 'implicit', 'explicit', or 'FAS'. The latter uses the Full Algebraic Storage (FAS) multigrid scheme, which can have implicit or explicit smoothers. & 'explicit' \\ \hline

polynomial order   & \textit{INTEGER}: Polynomial order to be assigned uniformly to all the elements of the mesh. If the keyword \textit{polynomial order file} is specified, the value of this keyword is overridden. & --* \\ \hline

polynomial order i \

polynomial order j \

polynomial order k & \textit{INTEGER}: Polynomial order in the i, j, or k component for all the elements in the domain. If used, the three directions must be declared explicitly, unless you are using a polynomial order file. If the keyword \textit{polynomial order file} is specified, the value of this keyword is overridden. & --* \\ \hline

polynomial order file  & \textit{CHARACTER}: Path to a file containing the polynomial order of each element in the domain. & --* \\ \hline

restart 			& \textit{LOGICAL}: If .TRUE., initial conditions of simulation will be read from restart file specified using the keyword \textit{restart file name}. & \textbf{Mandatory keyword} \\ \hline

cfl & \textit{REAL}: A constant related with the \textbf{convective} Courant-Friedrichs-Lewy (CFL) condition that the program will use to compute the time step size. & --** \\ \hline

dcfl & \textit{REAL}: A constant related with the \textbf{diffusive} Courant-Friedrichs-Lewy (DCFL) condition that the program will use to compute the time step size. & --** \\ \hline

dt  & \textit{REAL}: Constant time step size.  & --** \\ \hline

final time  & \textit{REAL}: This keyword is mandatory for time-accurate solvers & -- \\ \hline

mesh file name & \textit{CHARACTER}: Name of the mesh file. The currently supported formats are \textit{.mesh} (SpecMesh file format) and \textit{.h5} (HOPR hdf5 file format). & \textbf{Mandatory\ keyword} \\ \hline

mesh inner curves & \textit{LOGICAL}: Specifies if the mesh reader must suppose that the inner surfaces (faces connecting the elements of the mesh) are curved. This input variable only affects the hdf5 mesh reader. & .TRUE. \\ \hline

number of time steps & \textit{INTEGER}: \textit{Maximum} number of time steps that the program will compute.  & \textbf{Mandatory\ keyword} \\ \hline

output interval   & \textit{INTEGER}: In steady-state, this keyword indicates the interval of time steps to display the residuals on screen. In time-accurate simulations, this keyword indicates how often a 3D output file must be stored.  & \textbf{Mandatory\ keyword} \\ \hline

convergence tolerance & \textit{REAL}: Residual convergence tolerance for steady-state cases & \textbf{Mandatory\ keyword} \\ \hline

partitioning  & \textit{CHARACTER}: Specifies the method for partitioning the mesh in MPI simulations. Options are: 'metis' (the code must have been linked to METIS at compilation time, see Chapter~\ref{sect:compiling}), or 'SFC' (to use a space-filling curve method, no special compilation is needed for this option). & 'metis' \\ \hline

partitioning with weights & \textit{LOGICAL}: Specifies if the method for partitioning the mesh in MPI simulations takes the local polynomial order as weights. Necessary for local polynomial refinement & .TRUE. \\ \hline

manufactured solution & \textit{CHARACTER}: Must have the value '2D' or '3D'. When this keyword is used, the program will add source terms for the conservative variables taken into account an exact analytic solution for each primitive variable j ($\rho$, $u$, $v$, $w$, $p$) of the form:\

$j = j_C(1) + j_C(2) \sin(\pi j_C(5) x) + j_C(3) \sin(\pi j_C(6) y) + j_C(4) \sin(\pi j_C(7) z) $\

Where $j_C(i)$ are constants defined in the file \textit{ManufacturedSolutions.f90}. Proper initial and boundary conditions must be imposed (see the test case). The mesh must be a unit cube.
  & -- \\ \hline

\multicolumn{3}{p{16.4cm}}{*  One of these keywords must be specified} \\

\multicolumn{3}{p{16.4cm}}{** For Euler simulations, the user must specify either the CFL number or the time-step size. For Navier-Stokes simulations, the user must specify the CFL and DCFL numbers \textbf{or} the time-step size.}

\end{longtable}

\section{Boundary conditions}

The boundary conditions are specified as blocks in the control file. The block start with a the keywords `\#define' and ends with `\#end'. Inside the block the options are specified as a pair of keywords and values, just as the normal body of the rest of the file.

Each boundary condition can be individually defined or if multiple boundaries are set with the same definition, it could be done on the same block (with the name separated by a double under score `$\_$' sign). The name of each boundary must match with the one specified at the mesh file.

The block in general can be seen below. Table~\ref{tab:BC} show the values for the type keyword, and the possible value for the parameters depends on the boundary condition.

\begin{lstlisting}
#define boundary myBoundary1__myBoundary2__myBoundary3
	type        = typeValue
	parameter 1 = value_1
	parameter 2 = value_2
# end
\end{lstlisting}


\begin{table}[h]
\caption{Keywords for Boundary Conditions.}\label{tab:BC}

\begin{tabular}{|p{4cm}|p{10cm}|p{2.2cm}|}
\hline
\multicolumn{1}{|c|}{\textbf{Keyword}} & \multicolumn{1}{c|}{\textbf{Description}} & \multicolumn{1}{c|}{\textbf{Default value}} \\ \hline

type 	&
		\textit{CHARACTER}: Type of boundary condition to be applied. Options are: Inflow, Outflow, NoSlipWall, FreeSlipWall, Periodic, User-defined. &
							N/A \\ \hline

\end{tabular}
\end{table}

By default, NoSlipWall is adiabatic. Isothermal wall can be activated with the following block:

\begin{lstlisting}
#define boundary myBoundary1__myBoundary2__myBoundary3
	type = NoSlipWall
	wall type (adiabatic/isothermal) = isothermal
	wall temperature = 2000.0d0  !Wall temperature in K
# end
\end{lstlisting}

It is also possible to set a moving wall with the keyword wall velocity = \emph{value}. 

For periodic boundary conditions, the second boundary that must be used as a complement must be specified by the keyword `coupled boundary'. These two boundaries must have the same node position in all directions but one. For mesh files generated by commercial software where this strict rule is not imposed a comparison based on the minimum edge size of the face element can be used by a boolean parameter in the normal body of the control file (\emph{not in the block body}), with the keyword `periodic relative tolerance'.

\vspace{0.5 in}

%\emph{Juan's email (to be translated and adapted to the manual format as a complement):}
%
%Hola Gente,
%
%He tenido que hacer unas modificaciones bastante importantes en las BCs. Era la única parte del código que estaba “a la antigua” y no programada a objetos. Esto hacía que no fueran muy customizables, y por ejemplo las controlábamos con el número ese que siempre vale 0.0 jajaja. Ahora cada condición de contorno tiene los parámetros que necesitas y se pueden customizar. Lo malo es que ningún control file de los que tenéis van a seguir funcionando, pero os escribo los cambios para que sepáis adaptarlos, en cualquier caso, podéis pedirme ayuda y os cuento.
%
%Los cambios del código son:
%
%\begin{itemize}
%
%\item Las condiciones de contorno se definen igual que los monitores, con los $\#$define en la parte final del control file. Para definir una condicion de contorno se hace:
%\begin{lstlisting}
%        #define boundary name
%             type = Inflow/Outflow/NoSlipWall/FreeSlipWall/Periodic/User-defined
%             parametro1 = #valor
%             parametro2 = #valor
%        #end
%\end{lstlisting}
%\item Los parámetros1, … dependen de la condición de contorno que toque. Si no se especifica nada, pues está como estaba antes. Dos cambios importantes:
%
%          · He unificado las NoSlipWall (adiabatica e isoterma) en una sola. Por defecto es adiabática.
%          · En las periódicas es obligatorio ahora indicar a qué boundary se acopla (lo cual supone poco esfuerzo y reduce el tiempo de búsqueda al código)
%\begin{lstlisting}
%                 #define boundary name
%                       type = Periodic
%                      coupled boundary = nombredelboundaryalqueseacopla
%                 #end
%\end{lstlisting}
%\item Se pueden definir más de una condición de contorno del tirón, por ejemplo si boundary1, boundary2 y boundary3 son inflows se puede hacer:
%\begin{lstlisting}
%                 #define boundary boundary1__boundary2__boundary3
%                         type = Inflow
%                 #end
%\end{lstlisting}
%     es decir, separado con dos guiones bajos.
%
%\item Por pantalla, donde aparecía la info de las zones y tal, también aparece qué BC tiene y cuáles son los parámetros.
%
%\item La BC outflowspecifyP la llamo simplemente Outflow. Más que nada por que antes había algunos ficheros de control con la BC Outflow y no existía, pero por defecto se mandaba a Inflow. Para evitar problemas, pues Outflow.
%
%\item Los archivos de condición de contorno están en /physics/common en lugar de cada uno su archivo. Esto es por que al final son todas iguales y si se añade una nueva es más facil agregar un nuevo archivo que hacerlo individualmente en cada ecuación.
%
%\item Los bcTypeDictionary bcValueDictionary desaparecen. Las BC están en el module physics/common/BoundaryConditions.f90 como variable global, se llama BCs y dentro aloja todas las condiciones de contorno (una por zona, y en el mismo orden de las zonas).
%
%\end{itemize}
%
%Creo que eso es todo, lamento si os supone mucho cambio en vuestros ficheros de control que estéis corriendo a día de hoy, y si rompo algo que no reflejen los test. Pero estos cambios eran necesarios para darle más versatilidad (por ejemplo en multifase el inflow necesita bastante customización, para definir caudales de cada fase y cosas así). Además, creo que el enfoque OOP va en la dirección del resto del código.
%
%

%%%%%%%%%%%%%%%%%%%%%%%%%%%%%%%%%%%%%%%%%%%%%%%%%%%%%%%%%%%%%%%%%%%%%%%%%

\chapter{Spatial Discretization}

HORSES3D uses a high-order discontinuous Galerkin Spectral Element (DGSEM) method. As part of the numerical method, several options of volume and surface discretization options can be selected. Some options are only available in certain solvers.

\begin{longtable}{|p{4cm}|p{10cm}|p{2.2cm}|}
\caption{Keywords for spatial discretization.} \label{tab:spatialDisc} \\
\hline
\multicolumn{1}{|c|}{\textbf{Keyword}} & \multicolumn{1}{c|}{\textbf{Description}} & \multicolumn{1}{c|}{\textbf{Default value}} \\ \hline
\endfirsthead

\caption{Keywords for spatial discretization - continued.} \\
\hline
\multicolumn{1}{|c|}{\textbf{Keyword}} & \multicolumn{1}{c|}{\textbf{Description}} & \multicolumn{1}{c|}{\textbf{Default value}} \\ \hline
\endhead

Discretization nodes  	    & \textit{CHARACTER}: Node type to use in the solution. Options are:  
    \begin{itemize}
    \item Gauss
    \item Gauss-Lobatto
\end{itemize} 
                    & 'Gauss' \\ \hline

inviscid discretization	    & \textit{CHARACTER}: DG discretization of inviscid fluxes. Options are:
    \begin{itemize}
    \item Standard
    \item Split-Form
    \end{itemize} 
                    & 'Standard' \\ \hline

averaging & \textit{CHARACTER}: Type of averaging function to use in numerical fluxes and split-forms (if in use). Options are:
    \begin{itemize}
    \item Standard
    \item Morinishi
    \item Ducros
    \item Kennedy-Gruber
    \item Pirozzoli
    \item Entropy conserving
    \item Chandrasekar
    \item Skew-symmetric 1 (only for Incompressible Navier–Stokes)
    \item Skew-symmetric 2 (only for Incompressible Navier–Stokes)
    \end{itemize} 
                    & -- \\ \hline

viscous discretization & \textit{CHARACTER}: Method for viscous fluxes. Options are:
    \begin{itemize}
    \item BR1 (required for EULER RK3 hybrid temporal scheme)
    \item BR2
    \item IP
    \end{itemize} 
                    & 'BR2' \\ \hline

riemann solver      & \textit{CHARACTER}: Riemann solver for inviscid fluxes. Options are:
    \begin{itemize}
    \item Central
    \item Roe (Only for compressible Navier–Stokes)
    \item Standard Roe (Only for compressible Navier–Stokes)
    \item Roe-Pike (Only for compressible Navier–Stokes)
    \item Low dissipation Roe (Only for compressible Navier–Stokes)
    \item Lax-Friedrichs (Only for compressible and Incompressible Navier–Stokes)
    \item ES Lax-Friedrichs (Only for compressible Navier–Stokes, not including RANS)
    \item u-diss (Only for compressible Navier–Stokes, not including RANS)
    \item Rusanov (Only for compressible Navier–Stokes)
    \item Matrix dissipation (Only for compressible Navier–Stokes)
    \item Viscous NS (Only for compressible Navier–Stokes, not including RANS)
    \item Exact (Only for Incompressible Navier-Stokes, and Multiphase)
    \end{itemize} 
                    & \textbf{Mandatory keyword} \\ \hline

\end{longtable}

%%%%%%%%%%%%%%%%%%%%%%%%%%%%%%%%%%%%%%%%%%%%%%%%%%%%%%%%%%%%%%%%%%%%%%%%%

\chapter{Restarting a Case}

\begin{table}[htbp]
\caption{Keywords for restarting a case.}
\begin{tabular}{|l|p{10cm}|p{2.2cm}|}
\hline
\multicolumn{1}{|c|}{Keyword} & \multicolumn{1}{c|}{Description} & \multicolumn{1}{c|}{Default value} \\ \hline

restart 			& \textit{LOGICAL}: If .TRUE., initial conditions of simulation will be read from restart file specified using the keyword \textit{restart file name}. & \textbf{Mandatory keyword} \\ \hline

restart file name   & \textit{CHARACTER}: Name of the restart file to be written and, if keyword \textit{restart} = .TRUE., also name of the restart file to be read for starting the simulation. & \textbf{Mandatory\ keyword} \\ \hline

restart polorder & \textit{INTEGER}: Uniform polynomial order of the solution to restart from. This keyword is only needed when the restart solution is of a different order than the current case. & same as case's \\ \hline

restart nodetype & \textit{CHARACTER}: node type of the solution to restart from (\emph{Gauss} or \emph{Gauss-Lobatto}). This keyword is only needed when the restart node type is different than the current case. & same as case's \\ \hline

restart polorder file &
			\textit{CHARACTER}: File containing the polynomial orders of the solution to restart from. This keyword is only needed when the restart solution is of a different order than the current case.  &  same as case's\\ \hline

get discretization error of & \textit{CHARACTER}: Path to solution file. This can be used to estimate the discretization error of a solution when restarting from a higher-order solution. & -- \\ \hline

NS load from NSSA &
            \textit{LOGICAL}: Used only for NS simulations that are restarted from RANS SA. & .FALSE. \\ \hline

\end{tabular}
\label{tab:restartkey}
\end{table}

%%%%%%%%%%%%%%%%%%%%%%%%%%%%%%%%%%%%%%%%%%%%%%%%%%%%%%%%%%%%%%%%%%%%%%%%%

\chapter{Physics related keyword}

\section{Compressible flow}

\begin{table}[htbp]
\caption{Keywords for compressible flow (Euler / Navier-Stokes).}
\begin{tabular}{|l|p{10cm}|p{2.2cm}|}
\hline
\multicolumn{1}{|c|}{Keyword} & \multicolumn{1}{c|}{Description} & \multicolumn{1}{c|}{Default value} \\ \hline

Mach number					& \textit{REAL}:  & \textbf{Mandatory keyword} \\ \hline

Reynolds number				& \textit{REAL}:  & \textbf{Mandatory keyword} \\ \hline

Prandtl number				& \textit{REAL}:  & 0.72 \\ \hline

Turbulent Prandtl number	& \textit{REAL}:  & Equal to Prandtl \\ \hline

    AOA theta             & \textit{REAL}: Angle of attack (degrees), based on the spherical coordinates polar angle ($\theta$) definition  & 0.0 \\ \hline

    AOA phi               & \textit{REAL}: Angle of attack (degrees), based on the spherical coordinates azimuthal angle ($\varphi$) definition  & 0.0 \\ \hline

LES model					& \textit{CHARACTER(*)}: Options are:
\begin{itemize}
\item Vreman
\item Wale    
\item Smagorinsky
\item None
\end{itemize}
 & None \\ \hline

Wall model					& \textit{CHARACTER}:  & linear \\ \hline

\end{tabular}
\label{tab:compressibleFlowkey}
\end{table}

\subsection{Shock-capturing}

The shock-capturing module helps stabilize cases with discontinuous solutions, and may also improve the results of under-resolved turbulent cases. It is built on top of a \textit{Sensors} module that detects problematic flow regions, classifying them according to the value of the sensor,~$s$, mapped into the interval $a \in [0,1]$,
%
\begin{equation*}
    a = \left\{\begin{array}{ll}
        0, & \text{if } s \leq s_0 - \Delta s / 2, \\
        \frac{1}{2}\left[1+\sin\left(\frac{s-s_0}{\Delta s}\right)\right], & \text{if } s_0 - \Delta s / 2 < s < s_0 + \Delta s / 2,  \\
        1, & \text{elsewhere}.
    \end{array}\right.
\end{equation*}
%
The values of $s_0 = (s_1 + s_2)/2$ and~$\Delta s = s_2 - s_1$ depend on the sensor thresholds~$s_1$ and~$s_2$.

At the moment, flow regions where $a \leq 0$ are considered smooth and no stabilization algorithm can be imposed there. In the central region of the sensor, with $0 < a < 1$, the methods shown in the next table can be used and even scaled with the sensor value, so that their intensity increases in elements with more instabilities. Finally, the higher part of the sensor range can implement a different method from the table; however, the intensity is set to the maximum this time.

All the methods implemented introduce artificial dissipation into the equations, which can be filtered with an SVV kernel to reduce the negative impact on the accuracy of the solution. Its intensity is controlled with the parameters~$\mu$ (similar to the viscosity of the Navier-Stokes equations) and~$\alpha$ (scaling of the density-regularization term of the Guermond-Popov flux), which can be set as constants or coupled to the value of the sensor or to a Smagorinsky formulation.

\begin{longtable}{|l|p{10cm}|p{2.2cm}|}

\caption{Keywords for shock-capturing algorithms in the Navier-Stokes equations.} \label{tab:shockcapturingkey} \\
\hline
\multicolumn{1}{|c|}{Keyword} & \multicolumn{1}{c|}{Description} & \multicolumn{1}{c|}{Default value} \\ \hline
\endfirsthead

\caption{Keywords for shock-capturing algorithms in the Navier-Stokes equations -- continuation.} \\
\hline
\multicolumn{1}{|c|}{Keyword} & \multicolumn{1}{c|}{Description} & \multicolumn{1}{c|}{Default value} \\ \hline
\endhead

Enable shock-capturing &  \textit{LOGICAL}: Switch on/off the shock-capturing stabilization & .FALSE. \\ \hline
Shock sensor & \textit{CHARACTER}: Type of sensor to be used to detect discontinuous regions Options are:
    \begin{itemize}
        \item Zeros: always return 0
        \item Ones: always return 1
        \item Integral: integral of the sensor variable inside each element
        \item Integral with sqrt: square root of the Integral sensor
        \item Modal: based on the relative weight of the higher order modes
        \item Truncation error: estimate the truncation error of the approximation
        \item Aliasing error: estimate the aliasing error of the approximation
        \item GMM: clustering sensor based on the divergence of the velocity and the gradient of the pressure
    \end{itemize} & Integral \\ \hline
Shock first method & \textit{CHARACTER}: Method to be used in the middle region of the sensor ($a\in[0,1]$). Options are:
    \begin{itemize}
        \item None: Do not apply any smoothing
        \item Non-filtered: Apply the selected viscous flux without SVV filtering
        \item SVV: Apply an entropy-stable, SVV-filtered viscous flux
    \end{itemize} & None \\ \hline
Shock second method &\textit{CHARACTER}: Method to be used in the top-most region of the sensor ($a=1$). Options are:
    \begin{itemize}
        \item None: Do not apply any smoothing
        \item Non-filtered: Apply the selected viscous flux without SVV filtering
        \item SVV: Apply an entropy-stable, SVV-filtered viscous flux
    \end{itemize} & None \\ \hline
Shock viscous flux 1 & \textit{CHARACTER}: Viscous flux to be applied in the elements where~$a\in[0,1]$. Options are:
    \begin{itemize}
        \item Physical
        \item Guermond-Popov (only with entropy variables gradients)
    \end{itemize} & -- \\ \hline
Shock viscous flux 2 & \textit{CHARACTER}: Viscous flux to be applied in the elements where~$a=1$. Options are:
    \begin{itemize}
        \item Physical
        \item Guermond-Popov (only with entropy variables gradients)
    \end{itemize} & -- \\ \hline
Shock update strategy & \textit{CHARACTER}: Method to compute the variable parameter of the specified shock-captruing approach in the middle region of the sensor. Options are:
    \begin{itemize}
        \item Constant
        \item Sensor
        \item Smagorinsky: only for \textit{non-filtered} and \textit{SVV}
    \end{itemize} & Constant \\ \hline
Shock mu 1 & \textit{REAL/CHARACTER(*)}: Viscosity parameter~$\mu_1$, or~$C_s$ in the case of LES coupling & 0.0 \\ \hline
Shock alpha 1 & \textit{REAL}: Viscosity parameter~$\alpha_1$ & 0.0 \\ \hline
Shock mu 2 & \textit{REAL}: Viscosity parameter~$\mu_2$ & $\mu_1$ \\ \hline
Shock alpha 2 & \textit{REAL}: Viscosity parameter~$\alpha_2$ & $\alpha_1$ \\ \hline
Shock alpha/mu & \textit{REAL}: Ratio between~$\alpha$ and~$\mu$. It can be specified instead of~$\alpha$ itself to make it dependent on the corresponding values of~$\mu$, and it is compulsory when using LES coupling & -- \\ \hline
SVV filter cutoff & \textit{REAL/CHARACTER(*)}: Cutoff of the filter kernel,~$P$. If "automatic", its value is adjusted automatically & "automatic" \\ \hline
SVV filter shape & \textit{CHARACTER(*)}: Options are:
    \begin{itemize}
        \item Power
        \item Sharp
        \item Exponential
    \end{itemize} & Power \\ \hline
SVV filter type & \textit{CHARACTER(*)}: Options are:
    \begin{itemize}
        \item Low-pass
        \item High-pass
    \end{itemize} & High-pass \\ \hline
Sensor variable & \textit{CHARACTER(*)}: Variable used by the sensor to detect shocks. Options are:
    \begin{itemize}
        \item rho
        \item rhou
        \item rhov
        \item rhow
        \item u
        \item v
        \item w
        \item p
        \item rhop
        \item grad rho
        \item div v
    \end{itemize} & rhop \\ \hline
Sensor lower limit & \textit{REAL}: Lower threshold of the central sensor region,~$s_1$  & \textbf{Mandatory keyword} (except GMM) \\ \hline
Sensor higher limit & \textit{REAL}: Upper threshold of the central sensor region,~$s_2$ & \textbf{Mandatory keyword} (except GMM) \\ \hline
Sensor TE min N & \textit{INTEGER}: Minimum polynomial order of the coarse mesh used for the truncation error estimation & 1 \\ \hline
Sensor TE delta N & Polynomial order difference between the solution mesh and its coarser representation & 1 \\ \hline
Sensor TE derivative & \textit{CHARACTER*}: Whether the face terms must be considered in the estimation of the truncation error or not. Options are:
    \begin{itemize}
        \item Non-isolated
        \item Isolated
    \end{itemize} & Isolated \\ \hline
Sensor number of clusters & \textit{INTEGER}: Maximum number of clusters to use with the GMM sensor & 2 \\ \hline
Sensor min. steps & \textit{INTEGER}: Minimum number of time steps that an element will remain detected. The last positive value will be used if the sensor ``undetects'' an element too early & 1 \\ \hline
\end{longtable}

\subsubsection{Spectral Vanishing Viscosity}

The introduction of an SVV-filtered artificial flux helps dissipate high-frequency oscillations. The baseline viscous flux can be chosen as the Navier-Stokes viscous flux or the flux developed by Guermond and Popov. In any case, this flux is expressed in a modal base where it is filtered by any of the following three filter kernels:
%
\begin{itemize}
    \item power:~$\hat{F}^{\text{1D}}_i = (i/N)^P$,
    \item sharp:~$\hat{F}^{\text{1D}}_i = 0$ if~$i<P$,~$\hat{F}^{\text{1D}}_i = 1$ elsewhere,
    \item exponential:~$\hat{F}^{\text{1D}}_i = 0$ if~$i \leq P$, ~$\hat{F}^{\text{1D}}_i=\exp\left(-\frac{(i-N)^2}{(i-P)^2}\right)$ elsewhere.
\end{itemize}
%
The extension to three dimensions allows the introduction of two types of kernels based on the one-dimensional ones:
%
\begin{itemize}
    \item high-pass:~$\hat{F}^{\text{H}}_{ijk} = \hat{F}^{\text{1D}}_i \hat{F}^{\text{1D}}_j \hat{F}^{\text{1D}}_k$,
    \item low-pass:~$\hat{F}^{\text{L}}_{ijk} = 1 - \left(1-\hat{F}^{\text{1D}}_i\right)\left(1-\hat{F}^{\text{1D}}_j\right)\left(1-\hat{F}^{\text{1D}}_k\right)$,
\end{itemize}
%
being the low-pass one more dissipative and, thus, more suited to supersonic cases. The high-pass filter, on the other hand, works better as part of the SVV-LES framework for turbulent cases.

The cutoff parameter~$P$ can be set as "automatic", which uses a sensor to differentiate troubled elements from smooth regions. The stabilisation strategy then depends on the region:
%
\begin{itemize}
    \item smooth regions:~$P=4$,~$\mu=\mu_2$,~$\alpha=\alpha_2$,
    \item shocks:~$P=4$,~$\mu=\mu_1$,~$\alpha=\alpha_1$.
\end{itemize}

In addition to this, the viscosity~$\mu_1$ can be set to "Smagorinsky" to use the implemented SVV-LES approach. In this case, the~$\mu=\mu_{\text{LES}}$ viscosity is computed following a Smagorinsky formulation with~$C_s=\mu_2$ and the viscosity parameters do not depend on the region anymore,
%
\begin{equation*}
    \mu = C_s^2 \Delta^2|S|^2, \quad \alpha = \alpha_1.
\end{equation*}

\subsection{Acoustic}
The Ffowcs Williams and Hawkings (FWH) acoustic analogy is implemented as a complement to the compressible NS solver. It can run both during the execution of the NS (in-time) or as at a post-process step. The version implemented includes both the solid and permeable surface variations, but both of them for a static body subjected to a constant external flow, i.e. a wind tunnel case scenario. The specifications for the FWH are divided in two parts: the general definitions (including the surfaces) and the observers definitions. The former is detailed in Table~\ref{tab:FWHKey}, while the latter are defined in a block section, similar to the monitors (see Chapter~\ref{sect:monitors}):

\begin{lstlisting}
#define acoustic observer 1
   name     = SomeName
   position = [0.d0, 0.d0, 0.d0]
#end
\end{lstlisting}

To run the in-time computation, the observers must be defined in the control file. Beware that adding an additional observer will require to run the simulation again. To use the post-process computation, the solution on the surface must be saved at a regular time. Beware that it will need more storage. To run the post-process calculation the horses.tools binary is used, with a control file similar to the one use for the NS simulation (without monitors), and adding the keywords ''tool type'' and ''acoustic files pattern'', as explained in Table~\ref{tab:FWHKey}.

\begin{table}[htbp]
\caption{Keywords for acoustic analogy}
\begin{tabular}{|l|p{10cm}|p{2.2cm}|}
\hline
\multicolumn{1}{|c|}{Keyword} & \multicolumn{1}{c|}{Description} & \multicolumn{1}{c|}{Default value} \\ \hline

acoustic analogy                & \textit{CHARACTER(*)}: This is the main keyword for activating the acoustic analogy. The only options is: ''FWH''. & -- \\ \hline
acoustic analogy permeable      & \textit{LOGICAL}: Defines if uses a permeable or solid approach.  & .FALSE. \\ \hline
acoustic solid surface          & \textit{CHARACTER(*)}: Array containing the name of each boundary to be used as a surface for integration. In the form: '[bc1,bc2,bc3]'. Mandatory for using the solid surface variant. & -- \\ \hline
acoustic surface file           & \textit{CHARACTER(*)}: Path to a fictitious surface that will be used for integration. It must be tailor-made for the mesh. Mandatory for using the permeable surface variant. & -- \\ \hline
observers flush interval        & \textit{INTEGER}: Iteration interval to flush the observers information to the files. & 100 \\ \hline
acoustic solution save          & \textit{LOGICAL}: Defines whether it saves the NS solution on the surface. Mandatory for post-process computation. & .FALSE. \\ \hline
acoustic save timestep 	        & \textit{REAL}: Controls the time or iteration at which the FWH will be calculated (and saved if specified). If the key is missing it will be done at each timestep.  & -- \\ \hline
acoustic files pattern          & \textit{CHARACTER(*)}: Pattern to the path of all the saved solutions on the surface (To be used in horses.tools for the post-process calculation). & -- \\ \hline
tool type                       & \textit{CHARACTER(*)}: Necessary for post-process calculation. Defines the type of post-process of horses.tools. For the FWH analogy the value must be ''fwh post'' & -- \\ \hline
\end{tabular}
\label{tab:FWHKey}
\end{table}


\section{Incompressible Navier-Stokes}

Among the various incompressible Navier-Stokes models, {\fontfamily{qcr}\selectfont  HORSES3D} uses an artificial compressibility 
formulation, which converts the elliptic problem into a hyperbolic system of equations, at the expense of a non divergence–free 
velocity field. However, it allows one to avoid constructing an approximation that satisfies the inf–sup condition. 
This methodology is well suited for use as a fluid flow engine for interface–tracking multiphase flow models, as it allows the 
density to vary spatially. 

The artificial compressibility system of equations is
%
\begin{equation}
  \rho_{t} + \vec{\nabla}\cdot\left(\rho\vec{u}\right)=0 ,
  \label{eq:incomressible:continuity}
\end{equation}
%
\begin{equation}
\left(\rho\vec{u}\right)_t+\vec{\nabla}\cdot\left(\rho \vec{u}\vec{u}\right) = -\vec{\nabla}p + \vec{\nabla}\cdot\left(\frac{1}{\mathrm{Re}}\left(\vec{\nabla}\vec{u} + 
\vec{\nabla}\vec{u}^{T}\right)\right)+\frac{1}{\mathrm{Fr^{2}}}\rho\vec{e}_{g},
\label{eq:incomressible:momentum}
\end{equation}
%
\begin{equation}
  p_t + \rho_0 c_0^2 \vec{\nabla}\cdot\vec{u} = 0,
  \label{eq:incomressible:ACM}
\end{equation}

The factor $\rho_0$ is computed as $max(\rho_1/\rho_1,\rho_2/\rho_1)$.

This solver is run with the binary horses3d.ins. 

\begin{table}[htbp]
    \caption{Keywords for incompressible Navier-Stokes solver}
    \begin{tabular}{|l|p{10cm}|p{2.2cm}|}
    \hline
    \multicolumn{1}{|c|}{Keyword} & \multicolumn{1}{c|}{Description} & \multicolumn{1}{c|}{Default value} \\ \hline
    reference velocity (m/s)            & \textit{REAL}: Reference value for velocity                                                                                              & -- \\ \hline
    number of fluids (1/2)              & \textit{INTEGER}: Number of fluids present in the simulation                                                                             &  1 \\ \hline
    maximum density (kg/m\^{}3)            & \textit{REAL}: Maximum value used in the limiter of the density                                                                       &  Huge(1.0) \\ \hline
    minimum density (kg/m\^{}3)            & \textit{REAL}: Minimum value used in the limiter of the density                                                                       & -Huge(1.0) \\ \hline
    artificial compressibility factor   & \textit{REAL}: Artificial compressibility factor ${c_0}^2$.                                                                              & -- \\ \hline
    gravity acceleration (m/s\^{}2)        & \textit{REAL}: Value of gravity acceleration                                                                                          & -- \\ \hline
    gravity direction                   & \textit{REAL}: Array containing direction of gravity. Eg. $[0.0,-1.0,0.0]$.                                                              & -- \\ \hline
    \end{tabular}
    \label{tab:INSKey}
    \end{table}

The incompressible Navier Stokes solver has two modes: with 1 fluid and with 2 fluids. The required keywords are listed below.

    \begin{table}[htbp]
        \caption{Keywords for incompressible Navier-Stokes solver. Mode with 1 fluid}
        \begin{tabular}{|l|p{10cm}|p{2.2cm}|}
        \hline
        \multicolumn{1}{|c|}{Keyword} & \multicolumn{1}{c|}{Description} & \multicolumn{1}{c|}{Default value} \\ \hline
        density (kg/m\^{}3)            & \textit{REAL}: Density of the fluid   & -- \\ \hline
        viscosity (pa.s)            & \textit{REAL}: Viscosity of the fluid & -- \\ \hline                                                                        
    \end{tabular}
    \label{tab:INSKey_1fluid}
    \end{table}
    
    \begin{table}[htbp]
        \caption{Keywords for incompressible Navier-Stokes solver. Mode with 2 fluids}
        \begin{tabular}{|l|p{10cm}|p{2.2cm}|}
        \hline
        \multicolumn{1}{|c|}{Keyword} & \multicolumn{1}{c|}{Description} & \multicolumn{1}{c|}{Default value} \\ \hline
        fluid 1 density (kg/m\^{}3)           & \textit{REAL}: Density of the fluid 1 & -- \\ \hline
        fluid 1 viscosity (pa.s)           & \textit{REAL}: Viscosity of the fluid 1 & -- \\ \hline     
        fluid 2 density (kg/m\^{}3)           & \textit{REAL}: Density of the fluid 2 & -- \\ \hline
        fluid 2 viscosity (pa.s)           & \textit{REAL}: Viscosity of the fluid 2  & -- \\ \hline                                                                          
    \end{tabular}
    \label{tab:INSKey_2fluid}
    \end{table}    


\section{Multiphase}

The multiphase flow solver implemented in {\fontfamily{qcr}\selectfont  HORSES3D} is constructed by a combination of the diffuse interface model of Cahn–Hilliard
with the incompressible Navier–Stokes equations with variable density and artificial compressibility. 
This model is entropy stable and guarantees phase conservation with an accurate representation of surface tension effects. 
The modified entropy-stable version approximates:

\begin{equation}
c_t + \vec{\nabla}\cdot\left(c\vec{u}\right) = M_0 \vec{\nabla}^2 \mu,
\label{eq:governing:cahn--hilliard}
\end{equation}
%
\begin{equation}
\sqrt{\rho}\left(\sqrt{\rho}\vec{u}\right)_t+\vec{\nabla}\cdot\left(\frac{1}{2}\rho \vec{u}\vec{u}\right) 
+\frac{1}{2}\rho\vec{u}\cdot\vec{\nabla}\vec{u}+c\vec{\nabla}\mu
= -\vec{\nabla}p + \vec{\nabla}\cdot\left(\eta\left(\vec{\nabla}\vec{u} + 
\vec{\nabla}\vec{u}^{T}\right)\right)+\rho\vec{g},
\label{eq:governing:momentum-skewsymmetric-sqrtRho}
\end{equation}
%
\begin{equation}
  p_t + \rho_0 c_0^2 \vec{\nabla}\cdot\vec{u} = 0,
  \label{eq:governing:ACM}
\end{equation}
%
where $c$ is the phase field parameter, $M_0$ is the mobility, $\mu$ is the chemical potential, 
$\eta$ is the viscosity and $c_0$ is the artificial speed of sound. The factor $\rho_0$ is computed as $max(\rho_1/\rho_1,\rho_2/\rho_1)$.
Mobility $M_0$ is computed from the control file parameters chemical characteristic time $t_{CH}$, interface width $\epsilon$ and interface tension $\sigma$ with the formula 
$M_0 = L_{ref}^2 \epsilon /(t_{CH} \sigma)$. 

The term $M_0 \vec{\nabla}^2 \mu$ can be implicity integrated to reduce the stiffnes of the problem with the keyword time integration = IMEX. This is only recomended if the value of $M_0$ is very high so that the time step of the explicit scheme is very small. 

This solver is run with the binary horses3d.mu. 

\begin{table}[htbp]
    \caption{Keywords for multiphase Navier-Stokes solver}
    \begin{tabular}{|l|p{10cm}|p{2.2cm}|}
    \hline
    \multicolumn{1}{|c|}{Keyword} & \multicolumn{1}{c|}{Description} & \multicolumn{1}{c|}{Default value} \\ \hline
    fluid 1 density (kg/m\^{}3)           & \textit{REAL}: Density of the fluid 1 & -- \\ \hline
    fluid 1 viscosity (pa.s)           & \textit{REAL}: Viscosity of the fluid 1 & -- \\ \hline     
    fluid 2 density (kg/m\^{}3)           & \textit{REAL}: Density of the fluid 2 & -- \\ \hline
    fluid 2 viscosity (pa.s)           & \textit{REAL}: Viscosity of the fluid 2  & -- \\ \hline         
    reference velocity (m/s)            & \textit{REAL}: Reference value for velocity                                                                                              & -- \\ \hline                                           
    maximum density (kg/m\^{}3)            & \textit{REAL}: Maximum value used in the limiter of the density                                                                       &  Huge(1.0) \\ \hline
    minimum density (kg/m\^{}3)            & \textit{REAL}: Minimum value used in the limiter of the density                                                                       & -Huge(1.0) \\ \hline
    artificial compressibility factor   & \textit{REAL}: Artificial compressibility factor ${c_0}^2$.                                                                              & -- \\ \hline
    gravity acceleration (m/s\^{}2)        & \textit{REAL}: Value of gravity acceleration                                                                                          & -- \\ \hline
    gravity direction                   & \textit{REAL}: Array containing direction of gravity. Eg. $[0.0,-1.0,0.0]$.                                                              & -- \\ \hline
    velocity direction                   & \textit{REAL}: Array containing direction of velocity used for the outflow BC. Eg. $[1.0,0.0,0.0]$.                                                              & -- \\ \hline
    chemical characteristic time (s)    & \textit{REAL}: $t_{CH}$ controls the speed of the phase separation    & -- \\ \hline
    interface width (m)                 & \textit{REAL}: $\epsilon$ controls the interface width between the phases   & -- \\ \hline
    interface tension (N/m)             & \textit{REAL}: $\sigma$ controls the interface tension between the phases    & -- \\ \hline
    \end{tabular}
    \label{tab:MUKey}
    \end{table}

\section{Particles}

\boxed{\centering \textit{WARNING: Lagrangian particles are only implemented for the compressible Navier Stokes (horses.ns binary)}}\vspace{1ex}
\boxed{\centering \textit{WARNING: Lagrangian particles do not support MPI.}}\vspace{1ex}

{\fontfamily{qcr}\selectfont  HORSES3D} includes a two-way coupled Lagrangian solver.
Particles are tracked along their trajectories, according to the simplified particle equation of motion, where only contributions from Stokes drag and gravity are retained,
\begin{equation}
\label{eq:part_motion}
\frac{d y_i}{dt} = u_i, \quad \frac{d u_i}{dt} = \frac{v_i - u_i}{\tau_p} + g_i,
\end{equation}
where $u_i$ and $y_i$ are the \emph{ith} components of velocity and position of the particle, respectively. Furthermore, $v_i$ accounts for the continuous velocity of the fluid at the position of the particle.  We consider spherical Stokesian particles, so their mass and aerodynamic response time are $m_p = \rho_p \pi D_p^3/6$ and $\tau_p = \rho_p D_p^2 / 18\mu $, respectively, $\rho_p$ being the particle density and $D_p$ the particle diameter. 

Each particle is considered to be subject to a radiative heat flux $I_o$. The carrier phase is transparent to radiation, whereas the incident radiative flux on each particle is completely absorbed. Because we focus on relatively small volume fractions, the fluid-particle medium is considered to be optically thin. Under these hypotheses, the direction of the radiation is inconsequential, and each particle receives the same radiative heat flux, and its temperature $T_p$ is governed by
\begin{equation}
\label{eq:part_energy}
\frac{d}{dt} (m_p c_{V,p} T_p) = \frac{\pi D_p^2}{4} I_o - \pi D_p^2 h (T_p-T),
\end{equation}
where $c_{V,p}$ is the specific heat of the particle, which is assumed to be constant with respect to temperature. $T_p$ is the particle temperature and $h$ is the convective heat transfer coefficient, which for a Stokesian particle can be calculated from the Nusselt number $Nu = hD_p/k = 2$.

In practical simulations, integrating the trajectory of every particle is too expensive. Therefore, particles are agglomerated into parcels, each of them accounting for many particles with the same physical properties, position, velocity, and temperature. The evolution of the parcels is tracked with the same set of equations presented for the particles.

The two-way coupling means that fluid flow is modified because of the presence of particles. Therefore, the Navier-Stokes equations are enriched with the following source terms:
%
\begin{equation}
\boldsymbol{S} = \beta\left[\begin{array}{c} 0 \\
                                                                       \sum_{n=1}^{N_p} \frac{m_p}{\tau_p} (u_{1,n}-v_1)\delta(\mathbf{x} - \mathbf{y}_n)  \\
                                                                       \sum_{n=1}^{N_p} \frac{m_p}{\tau_p} (u_{2,n}-v_2)\delta(\mathbf{x} - \mathbf{y}_n) \\
                                                                      \sum_{n=1}^{N_p} \frac{m_p}{\tau_p} (u_{3,n}-v_3)\delta(\mathbf{x} - \mathbf{y}_n) \\
                                                                      \sum_{n=1}^{N_p} \pi D_p^2 h (T_{p,n} - T) \delta(\mathbf{x} - \mathbf{y}_n )
\end{array}\right],
\label{eq:particlessource}
\end{equation}
%
where $\delta$ is the Dirac delta function, $N_p$ is the number of parcels, $\beta$ is the number of particles per parcel and $u_{i,n}$, $\mathbf{y}_{i,n}$, $T_{p,n}$ are the velocity, spatial coordinates, and temperature of the parcel \emph{nth}.
The dimensionless form of the Navier Stokes equations can be seen in the appendix at the end of this document. 

Particles are solved in a box domain inside the flow domain. The box is defined with the keywords ``minimum box'' and ``maximum box''. The boundary conditions for the particles are defined with the keyword ``bc box''. Possible options are inflow/outflow, periodic and wall (with perfect rebound). 

\begin{table}[htbp]
    \caption{Keywords for particles Navier-Stokes solver}
    \begin{tabular}{|l|p{10cm}|p{2.2cm}|}
    \hline
    \multicolumn{1}{|c|}{Keyword} & \multicolumn{1}{c|}{Description} & \multicolumn{1}{c|}{Default value} \\ \hline
    lagrangian particles                & \textit{LOGICAL}: If .true. activates particles & .false. \\ \hline
    stokes number                       & \textit{REAL}: Stokes number which for Stokesian particles is $St=\frac{\rho_p D_p^2 u_o}{18 L_o \mu_o}$  & -- \\ \hline
    Gamma                               & \textit{REAL}: Ratio between specific heat of particles and fluid $\Gamma=c_{v,p}/c_{v}$ & -- \\ \hline
    phi\_m                               & \textit{REAL}: Ratio between total mass of particles and fluid $\phi_m=\frac{m_p N_p}{\rho_o L_o^3}$  & -- \\ \hline
    Radiation source                    & \textit{REAL}: Non dimensional radiation source intensity $I_o^*=\frac{I_o D_p}{4k_oT_o}$   & -- \\ \hline
    Froude number                       & \textit{REAL}: Froude number $Fr=\frac{u_o}{\sqrt{g_o L_o}}$  & -- \\ \hline
    high order particles source term    & \textit{LOGICAL}: source term with high order dirac delta or averaged in the whole element  & .false. \\ \hline
    number of particles                 & \textit{INTEGER}: Total number of parcels in the simulation & -- \\ \hline
    particles per parcel                & \textit{REAL}: $\beta$ particles per parcel & -- \\ \hline

    \end{tabular}
    \label{tab:PartKey1}
    \end{table}

    \begin{table}[htbp]
    \caption{Keywords for particles Navier-Stokes solver - continued}
    \begin{tabular}{|l|p{10cm}|p{2.2cm}|}
    \hline
    \multicolumn{1}{|c|}{Keyword} & \multicolumn{1}{c|}{Description} & \multicolumn{1}{c|}{Default value} \\ \hline

    Gravity direction                   & \textit{INTEGER}: Array with direction of gravity. Only required if Fr number is specified.[0,0,-1] & -- \\ \hline
    particles file                      & \textit{CHARACTER(*)}: Path to file with initial position of the particles. If not provided, initialization without particles inside the domain. Not compatible with injecting particles in the domain (see keyword below). & -- \\ \hline
    vel and temp from file              & \textit{LOGICAL}: If .true. Initial velocity and temperature of particles read from file. If .false., initial velocity and temperature of particles are the same as flow at the position of the particle. & -- \\ \hline
    injection                           & \textit{LOGICAL}: If .true. injection of particles through a face of the box. & -- \\ \hline
    particles injection                 & \textit{INTEGER}: Array with a vector indicating the direction of the injection. Eg., [0,1,0]      & -- \\ \hline
    particles per step                  & \textit{INTEGER}: Number of particles injected per time step. & -- \\ \hline
    particles iter period               & \textit{INTEGER}: Iteration period for particles injection. Set to 1 to inject particles every time step.   & -- \\ \hline
    particles injection velocity        & \textit{REAL}:   Array with particles injection non dimensional velocity. Eg., [0.d0,1.d0,0.d0] & -- \\ \hline
    particles injection temperature     & \textit{REAL}:  Particles injection non-dimensional temperature & -- \\ \hline
    minimum box                         & \textit{REAL}: Array with minimum x,y,z coordinates of box with particles. Eg., [0.d0,0.d0,0.d0]  & -- \\ \hline
    maximum box                         & \textit{REAL}: Array with maximum x,y,z coordinates of box with particles. Eg., [4.d-2,1.6d-1,4.d-2]  & -- \\ \hline
    bc box                              & \textit{INTEGER}: Array with boundary conditions for particles box in the form [yz,xz,xy]. 
    0 is inflow/outflow, 1 is wall, 2 is periodic. Eg.,   [2,0,2]. In this example planes yz and xy are periodic, while plane xz is inflow/outflow for particles. & -- \\ \hline

    \end{tabular}
    \label{tab:PartKey2}
    \end{table}


\section{Complementary Modes}

\subsection{Wall Function}

The wall function overwrites the viscous flux on the specified boundaries based on an specific law using a Newman condition. It must be used as a complement of no slip boundary condition. Table~\ref{tab:wallFunckey} shows the parameters that can be set in the control file. The frictional velocity is calculated using the instantaneous values of the first node (either Gauss or Gauss-Lobatto) of the element neighbour of the face element (at the opposite side of the boundary face). Currently is only supported for the compressible Navier-Stokes solver.

The standard wall function uses the Reichardt law, solving the algebraic non-linear equation using the newton method to obtain the frictional velocity. The ABL function uses the logarithmic atmospheric boundary layer law, using the aerodynamic roughness; the frictional velocity is without using any numerical method.

\begin{table}[htbp]
\caption{Keywords for Wall Function}
\begin{tabular}{|l|p{10cm}|p{2.2cm}|}
\hline
\multicolumn{1}{|c|}{Keyword} & \multicolumn{1}{c|}{Description} & \multicolumn{1}{c|}{Default value} \\ \hline

Wall Function					& \textit{CHARACTER(*)}: This is the main keyword for activating the wall function. Identifies the wall law to be used. Options are:
    \begin{itemize}
    \item Standard: uses the Reichardt law.
    \item ABL: uses the atmospheric boundary layer law.
    \end{itemize}
& -- \\ \hline

Wall Function Boundaries 	    & \textit{CHARACTER(*)}: Array containing the name of each boundary to be used. In the form: '[bc1,bc2,bc3]'. Mandatory for using the wall function. & -- \\ \hline

Wall Function kappa		    	& \textit{REAL}: von Karman constant  & 0.38 \\ \hline
Wall Function C     			& \textit{REAL}: Log law 'C' constant  & 4.1 \\ \hline
Wall Function Seed		    	& \textit{REAL}: Initial value for the newton method  & 1.0 \\ \hline
Wall Function Damp		    	& \textit{REAL}: Initial value damp for the newton method & 1.0 \\ \hline
Wall Function Tolerance			& \textit{REAL}: Tolerance for the newton method  & $10^{-10}$ \\ \hline
Wall Function max iter			& \textit{INTEGER}: Maximum number of iterations for the newton method  & 100 \\ \hline
Wall Roughness	        		& \textit{REAL}: Aerodynamic roughness for the ABL wall function. Mandatory value for the ABL law.  & -- \\ \hline
Wall Plane Displacement			& \textit{REAL}: Plane displacement due to roughness for the ABL wall function  & 0.0 \\ \hline
Wall Function Use Average       & \textit{LOGICAL}: Use the time average of the velocity in the wall function, each time step the time average is recalculated. & .FALSE. \\ \hline


\end{tabular}
\label{tab:wallFunckey}
\end{table}

\subsection{Tripping}

A numerical source term is added to the momentum equations to replicate the effect of a tripping mechanism used commonly in explerimental tests. The forcing is described via the product of two independent functions: one that depends streamwise and vertical directions (space only) and the other one describing the spanwise direction and time (space and time).
It can be used for the compressible NS, both LES and RANS.
The keywords for the trip options are listed in table \ref{tab:trip}.

\begin{table}[htbp]
\caption{Keywords for Tripping model}
\begin{tabular}{|l|p{10cm}|p{2.2cm}|}
\hline
\multicolumn{1}{|c|}{Keyword} & \multicolumn{1}{c|}{Description} & \multicolumn{1}{c|}{Default value} \\ \hline

use trip                        & \textit{LOGICAL}: This is the main keyword for activating the trip & .FALSE. \\ \hline
trip time scale                 & \textit{REAL}: Time interval between the change of the time dependent part of the trip. & \textbf{Mandatory} \\ \hline
trip number of modes            & \textit{INTEGER}: Number of Fourier modes in the spanwise direction of the trip. & \textbf{Mandatory} \\ \hline
trip z points                   & \textit{INTEGER}: Number of points to create the Fourier Transformation of the spanwise direction, it must be greater than the number of modes and should be ideally equal to the number of discretization points of the mesh in the same direction. & \textbf{Mandatory} \\ \hline
trip attenuation                & \textit{REAL ARRAY(2)}: Length scale of the gaussian attenuation of the trip, the first position is the streamwise direction and the second is the wall-normal direction. & \textbf{Mandatory} \\ \hline
trip zone                       & \textit{CHARACTER(*) ARRAY(:)}: Boundary condition name that constrains at least one surface where the trip center is located. It can be ither one or two boundary conditions, the latter used to generate a trip in two different positions (i.e. pressure and suction sides of an airfoil).  & \textbf{Mandatory} \\ \hline
trip center                     & \textit{REAL}: Position of the origin of the trip in the streamwise direction. & \textbf{Mandatory} \\ \hline
trip center 2                   & \textit{REAL}: Position of the origin of the second trip, if used, in the streamwise direction. & -- \\ \hline
trip amplitude                  & \textit{REAL}: Maximum time varying amplitude of the trip. & 1.0 \\ \hline
trip amplitude steady           & \textit{REAL}: Maximum steady amplitude of the trip. & 0.0 \\ \hline
random seed 1                   & \textit{INTEGER}: Number used to initialize the random number generator of the trip. It can vary in different simulations but must remain constant for a restart. & 930187532 \\ \hline
random seed 2                   & \textit{INTEGER}: Number used to initialize the random number generator of the trip. It can vary in different simulations but must remain constant for a restart. & 597734650 \\ \hline
\end{tabular}
\label{tab:trip}
\end{table}


\chapter{Implicit Solvers with Newton linearisation}
\section{General Keywords}
The keywords for the implicit solvers are listed in table \ref{tab:implicitkey}

%\begin{table}[htbp]
%\caption{Keywords for implicit solvers.}
%\begin{tabular}{|l|p{10cm}|p{2.2cm}|}
%\hline
%\multicolumn{1}{|c|}{Keyword} & \multicolumn{1}{c|}{Description} & \multicolumn{1}{c|}{Default value} \\ \hline

\begin{longtable}{|p{4cm}|p{10cm}|p{2.2cm}|}
\caption{Keywords for implicit solvers.} \label{tab:implicitkey} \\
\hline
\multicolumn{1}{|c|}{\textbf{Keyword}} & \multicolumn{1}{c|}{\textbf{Description}} & \multicolumn{1}{c|}{\textbf{Default value}} \\ \hline
\endfirsthead

\caption{Keywords for implicit solvers - continued.} \\
\hline
\multicolumn{1}{|c|}{\textbf{Keyword}} & \multicolumn{1}{c|}{\textbf{Description}} & \multicolumn{1}{c|}{\textbf{Default value}} \\ \hline
\endhead

\textbf{time integration} & \textit{CHARACTER}: This is the main keyword for activating the implicit solvers. The value of it should be set to 'implicit' for the BDF solvers and to 'rosenbrock' for Rosenbrock schemes. & 'explicit' \\ \hline

linear solver           & \textit{CHARACTER}: Specifies the linear solver that has to be used. Options are:\
				\begin{itemize}
					\item 'petsc': PETSc library Krylov-Subspace methods. Available in serial, but use with care (PETSc is not thread-safe, so OpenMP is not recommended). Only available in parallel (MPI) for preallocated Jacobians (see next section).
					\item 'pardiso': Intel MKL PARDISO. Only available in serial or with OpenMP.
					\item 'matrix-free gmres': A matrix-free version of the GMRES algorithm. Can be used without preconditioner or with a recursive GMRES preconditioner using 'preconditioner=GMRES'. Available in serial and parallel (OpenMP+MPI)
					\item 'smooth': Traditional iterative methods. One can select either 'smoother=WeightedJacobi' or 'smoother=BlockJacobi'.
					\item 'matrix-free smooth': A matrix-free version of the previous solver. Only available with 'smoother=BlockJacobi'.
				\end{itemize}
										& 'petsc'  \\ \hline



\end{longtable}

\section{Keywords for the BDF Methods}

The BDF methods implemented in HORSES3D use a Newton's method

\begin{longtable}{|p{4cm}|p{9cm}|p{3.2cm}|}
\caption{Keywords for the BDF solvers.} \label{tab:BDFkey} \\
\hline
\multicolumn{1}{|c|}{\textbf{Keyword}} & \multicolumn{1}{c|}{\textbf{Description}} & \multicolumn{1}{c|}{\textbf{Default value}} \\ \hline
\endfirsthead

\caption{Keywords for the BDF solvers - continued.} \\
\hline
\multicolumn{1}{|c|}{\textbf{Keyword}} & \multicolumn{1}{c|}{\textbf{Description}} & \multicolumn{1}{c|}{\textbf{Default value}} \\ \hline
\endhead

bdf order             & \textit{INTEGER}: If present, the solver uses a BDF solver of the specified order. BDF1 - BDF5 are available, and BDF2 - BDF5 require constant time steps. & 1 \\ \hline

jacobian by convergence & \textit{LOGICAL}: When .TRUE., the Jacobian is only computed when the convergence falls beneath a threshold (hard-coded). This improves performance.  & .FALSE. \\ \hline

compute jacobian every & \textit{INTEGER}: Forces the Jacobian to be computed in an interval of iterations that is specified. & Inf \\ \hline

%% Keywords for the Newton's method

print newton info       & \textit{LOGICAL}: If .TRUE., the information of the Newton iterations will be displayed. &  '.FALSE.'\\ \hline
implicit adaptive dt  & \textit{LOGICAL}: Specifies if the time-step should be computed according to the convergence behavior of the Newton iterative method and the linear solver. & .FALSE. \\ \hline

newton tolerance   & \textit{REAL}: Specifies the tolerance for the Newton's method. &  $10^{-6}$ for time-accurate simulations, or $MaxResidual \times a$ for steady-state simulations, where $a$ is the keyword \textit{newton factor} \\ \hline

newton max iter    & \textit{INTEGER}: Maximum number of Newton iterations for BDF solver.  & 30 \\ \hline

linsolver max iter	& \textit{INTEGER}: Maximum number of iterations to be taken by the linear solver. This keyword only affects iterative linear solvers.		    & 500 \\ \hline

newton factor 	& \textit{REAL}: In simulations that are not time-accurate, the tolerance of the Newton's method is a function of the residual: $MaxResidual \times a$, where $a$ is the specified value.	& $10^{-3}$	\\ \hline

%% Keywords for the linear solver tolerance

linsolver tol factor 	& \textit{REAL}: The linear solver tolerance is a function of the absolute error of the Newton's method: $tol=\norm{e}_{\infty}*a^i$, where $e$ is the absolute error of the Newton's method, $i$ is the Newton iteration number, and $a$ is the specified value.	& $0.5$	\\ \hline

newton first norm  & \textit{REAL}: Specifies an assumed infinity norm of the absolute error of the Newton's method at the iteration $0$ of the time step $1$.
This can change the behavior of the first Newton iterative method because of the dependency of the linear system tolerance on the absolute error of the Newton's method (see keyword \textit{linsolver tol factor}).
				   & $0.2$ \\ \hline







\end{longtable}

\section{Keywords for the Rosenbrock-Type Implicit Runge-Kutta Methods}

\begin{longtable}{|p{4cm}|p{10cm}|p{2.2cm}|}
\caption{Keywords for the Rosenbrock schemes.} \label{tab:Rosenbrockkey} \\
\hline
\multicolumn{1}{|c|}{\textbf{Keyword}} & \multicolumn{1}{c|}{\textbf{Description}} & \multicolumn{1}{c|}{\textbf{Default value}} \\ \hline
\endfirsthead

\caption{Keywords for the Rosenbrock schemes - continued.} \\
\hline
\multicolumn{1}{|c|}{\textbf{Keyword}} & \multicolumn{1}{c|}{\textbf{Description}} & \multicolumn{1}{c|}{\textbf{Default value}} \\ \hline
\endhead

rosenbrock scheme	& \textit{CHARACTER}: Rosenbrock scheme to be used. Currently, only the \textit{RO6-6} is implemented.		    & -- \\ \hline

\end{longtable}

\section{Jacobian Specifications}
The Jacobian must be defined using a block of the form:
\begin{lstlisting}
#define Jacobian
   type = 2
   print info = .TRUE.
   preallocate = .TRUE.
#end
\end{lstlisting}

\begin{longtable}{|p{4cm}|p{10cm}|p{2.2cm}|}
\caption{Keywords for Jacobian definition block.} \label{tab:Jacobiankey} \\
\hline
\multicolumn{1}{|c|}{\textbf{Keyword}} & \multicolumn{1}{c|}{\textbf{Description}} & \multicolumn{1}{c|}{\textbf{Default value}} \\ \hline
\endfirsthead

\caption{Keywords for Jacobian definition block - continued.} \\
\hline
\multicolumn{1}{|c|}{\textbf{Keyword}} & \multicolumn{1}{c|}{\textbf{Description}} & \multicolumn{1}{c|}{\textbf{Default value}} \\ \hline
\endhead


type             & \textit{INTEGER}: Specifies the type of Jacobian matrix to be computed. Options are:\
				\begin{enumerate}
					\item Numerical Jacobian: Uses a coloring algorithm and a finite difference method to compute the DG Jacobian matrix (only available with shared memory parallelization).
					\item Analytical Jacobian: Available with shared (OpenMP) or distributed (MPI) memory parallelization for advective and/or diffusive nonlinear conservation laws, \textbf{BUT} only for the standard DGSEM (no split-form).
					\end{enumerate}
										& \textbf{Mandatory Keyword} \\ \hline

print info      & \textit{LOGICAL}: Specifies the verbosity of the Jacobian subroutines  & .TRUE. \\ \hline

preallocate     & \textit{LOGICAL}: Specifies if the Jacobian must be allocated in preprocessing (.TRUE. - only available for advective/diffusive nonlinear conservation laws) or every time it is computed (.FALSE.)  & .FALSE. \\ \hline

\end{longtable}

%%%%%%%%%%%%%%%%%%%%%%%%%%%%%%%%%%%%%%%%%%%%%%%%%%%%%%%%%%%%%%%%%%%%%%%%%%

\chapter{Explicit Solvers}

Explicit time integration schemes available in HORSES3D.
The main keywords to use it are shown in Table \ref{tab:explicitKey}.

\begin{longtable}{|p{4cm}|p{10cm}|p{2.2cm}|}
\caption{Keywords for the multigrid solver.} \label{tab:explicitKey} \\
\hline
\multicolumn{1}{|c|}{\textbf{Keyword}} & \multicolumn{1}{c|}{\textbf{Description}} & \multicolumn{1}{c|}{\textbf{Default value}} \\ \hline
\endfirsthead

\caption{Keywords for the multigrid solver - continued.} \\
\hline
\multicolumn{1}{|c|}{\textbf{Keyword}} & \multicolumn{1}{c|}{\textbf{Description}} & \multicolumn{1}{c|}{\textbf{Default value}} \\ \hline
\endhead

\textbf{time integration} & \textit{CHARACTER}: This is the main keyword to activate the multigrid solvers. The value of it should be set to 'FAS' for the Full Approximation Scheme (FAS) nonlinear multigrid  solvers and to 'AnisFAS' for anisotropic FAS schemes. & 'explicit' \\ \hline

\textbf{simulation type} & \textit{CHARACTER}: Specifies if HORSES3D must perform a ’steady-state’ or a ’time-accurate’. If 'time-accurate' the solver switches to BDF integration and uses FAS as a pseudo problem solver. Compatible only with 'FAS'. & 'steady-state' \\ \hline

explicit method & \textit{CHARACTER}: Select desired Runge-Kutta solver. Options are: 'Euler', 'RK3', 'RK5', 'RKOpt', 'SSPRK33',  'SSPRK43' and \, 'EULER RK3'. The option 'EULER RK3' is a hybrid temporal scheme thought to be used with RL p-Adaptation and BR1 as viscous discretization. & RK3 \\ \hline

rk order & \textit{INTEGER}: Order of Runge-Kutta method optimized for steady-state solver ('RKOpt'). Possible orders are from 2 to 7. & 2 \\ \hline

limit timestep & \textit{LOGICAL}: Activate the possitivity limiter of Zhang and Shu (only for SSPRK methods). & .false. \\ \hline

limiter minimum & \textit{REAL}: Minimum value of density and pressure allowed by the limiter. & 1e-13 \\ \hline

\end{longtable}

%%%%%%%%%%%%%%%%%%%%%%%%%%%%%%%%%%%%%%%%%%%%%%%%%%%%%%%%%%%%%%%%%%%%%%%%%


\chapter{Nonlinear $p$-Multigrid solver (FAS)}

The code has an implementation of the Full Approximation Scheme (FAS) nonlinear $p$-multigrid method. The main keywords to use it are shown in Table \ref{tab:multigridKey}.

\begin{longtable}{|p{4cm}|p{10cm}|p{2.2cm}|}
\caption{Keywords for the multigrid solver.} \label{tab:multigridKey} \\
\hline
\multicolumn{1}{|c|}{\textbf{Keyword}} & \multicolumn{1}{c|}{\textbf{Description}} & \multicolumn{1}{c|}{\textbf{Default value}} \\ \hline
\endfirsthead

\caption{Keywords for the multigrid solver - continued.} \\
\hline
\multicolumn{1}{|c|}{\textbf{Keyword}} & \multicolumn{1}{c|}{\textbf{Description}} & \multicolumn{1}{c|}{\textbf{Default value}} \\ \hline
\endhead

\textbf{time integration} & \textit{CHARACTER}: This is the main keyword to activate the multigrid solvers. The value of it should be set to 'FAS' for the Full Approximation Scheme (FAS) nonlinear multigrid  solvers and to 'AnisFAS' for anisotropic FAS schemes. & 'explicit' \\ \hline

\textbf{simulation type} & \textit{CHARACTER}: Specifies if HORSES3D must perform a ’steady-state’ or a ’time-accurate’. If 'time-accurate' the solver switches to BDF integration (the exact method can be set using 'bdf order' option) and uses FAS as a local steady-state problem solver. Compatible only with 'FAS'. & 'steady-state' \\ \hline

multigrid levels & \textit{INTEGER}: Number of multigrid levels for the computations. & \textbf{Mandatory keyword} \\ \hline

delta n          & \textit{INTEGER}: Interval of reduction of polynomial order for creating coarser multigrid levels.& 1 \\ \hline

multigrid output & \textit{LOGICAL}: If .TRUE., the residuals at the different multigrid levels will be displayed. & .FALSE. \\ \hline

mg sweeps    & \textit{INTEGER}: Number of smoothing sweeps to be taken. & 1* \\ \hline

mg sweeps pre    & \textit{INTEGER}: Number of pre-smoothing sweeps to be taken. & 1* \\ \hline

mg sweeps post    & \textit{INTEGER}: Number of post-smoothing sweeps to be taken. & 1* \\ \hline

mg sweeps coarsest   & \textit{INTEGER}: Number of pre- and post-smoothing sweeps to be taken on the coarsest multigrid level. & Average between pre-sweeps and post-sweeps \\ \hline

mg sweeps exact & \textit{INTEGER(:)}: Alternative to 'mg sweeps'. Defines exact number of pre- and post- smoothing sweeps to be taken on each level. Index of the array indicates the MG level for the sweeps to be performed, e.g. [1,4] performs 1 pre-sweep and 1 post-sweep on level 1 and 4 pre-\/post-sweeps on level 2. & 1* \\ \hline

mg sweeps pre exact & \textit{INTEGER(:)}: Alternative to 'mg sweeps pre'. Defines exact number of pre-smoothing sweeps to be taken on each level. Index of the array indicates the MG level for the sweeps to be performed, e.g. [1,4] performs 1 pre-sweep on level 1 and 4 pre-sweeps on level 2. & 1* \\ \hline

mg sweeps post exact & \textit{INTEGER(:)}: Alternative to 'mg sweeps post'. Defines exact number of post-smoothing sweeps to be taken on each level. Index of the array indicates the MG level for the sweeps to be performed, e.g. [1,4] performs 1 post-sweep on level 1 and 4 post-sweeps on level 2. & 1* \\ \hline

mg smoother     & \textit{CHARACTER}: The smoothing technique to be used. The keywords and possible explicit smoothers are the same as the 'explicit method' in \ref{tab:explicitKey}. For the semi-implicit residual relaxation use 'BIRK5'. & RK3 \\ \hline

%% FMG keywords

fasfmg residual & \textit{REAL}: When this keyword is used, the code uses a full multigrid (FMG) method to obtain an initial condition for the simulation.
The initial condition has the specified residual.	& --	\\ \hline

fasfmg save solutions & \textit{LOGICAL}: Save the solutions that are obtained at the different FMG levels.
Only usable when \textit{fasfmg residual} is used. 	& .FALSE.	\\ \hline

%% Smoothing tuning keywords

postsmooth option    & \textit{CHARACTER}: When this keyword is used, the code performs extra post-smoothing sweeps, so that the final residual after completing the post-smoothing is lower than the residual achieved by the pre-smoothing. The options are:\

\begin{itemize}
\item \textit{f-cycle}: Do the extra post-smoothing with an FMG cycle.
\item \textit{smooth}: Do normal smoothing.
\end{itemize}  & -- \\ \hline

smooth fine & \textit{REAL}: Extra pre-smoothing is performed on a multigrid level of order $P$, until a residual is obtained $\norm{\tilde{\Res}^{P}}_{\infty} < \eta \norm{\tilde{\Res}^{N}}_{\infty}$, where $N$ is the polynomial order of the next (coarsest) grid, and $\eta$ is the specified value.	& --	\\ \hline

max mg sweeps & \textit{INTEGER}: Maximum number of smoothing sweeps to be performed. This only makes sense if one uses the keywords \textit{postsmooth option} and/or \textit{smooth fine}.	& 10000	\\ \hline

mg initialization & \textit{LOGICAL}: Sets the initial explicit residual smoothing with RK3 and local time stepping. & .FALSE.	\\ \hline

initial residual & \textit{REAL}: Threshold for the $\norm{\tilde{\Res}^{P}}_{\infty}$ after which solver switches from the 'mg initialization' settings to user specified. & 1.0	\\ \hline

initial cfl & \textit{REAL}: CFL and DCFL number for initial residual smoothing. & 0.1	\\ \hline

\multicolumn{3}{p{16.4cm}}{*  The user must specify \textit{mg sweeps pre} \textbf{and} \textit{mg sweeps post}, or \textit{mg sweeps}.} \\

\end{longtable}

%%%%%%%%%%%%%%%%%%%%%%%%%%%%%%%%%%%%%%%%%%%%%%%%%%%%%%%%%%%%%%%%%%%%%%%%%
\chapter{p-Adaptation Methods}
The p-adaptation methods are used when the p-adaptation region is specified in the control file. There are two different types of p-adaptation algorithms: A Truncation Error (TE) algorithm and a Reinforcement Learning (RL) algorithm.

\begin{longtable}{|p{4cm}|p{10cm}|p{2.2cm}|}
\caption{General keywords for the p-adaptation algorithms.} \label{tab:pAdaptationKey} \\
\hline
\multicolumn{1}{|c|}{\textbf{Keyword}} & \multicolumn{1}{c|}{\textbf{Description}} & \multicolumn{1}{c|}{\textbf{Default value}} \\ \hline
\endfirsthead

\caption{General keywords for the p-adaptation algorithms - continued.} \\
\hline
\multicolumn{1}{|c|}{\textbf{Keyword}} & \multicolumn{1}{c|}{\textbf{Description}} & \multicolumn{1}{c|}{\textbf{Default value}} \\ \hline
\endhead

adaptation type & \textit{CHARACTER}: Can be either "TE" or "RL". & TE \\ \hline

Nmax          & \textit{INTEGER}(3): Maximum polynomial order in each direction for the p-adaptation algorithm (limited to 6 for RL adaptation type). &
					\textbf{Mandatory keyword} \\ \hline

Nmin          & \textit{INTEGER}(3): Minimum polynomial order in each direction for the p-adaptation algorithm. &
					[1,1,1] \\ \hline

conforming boundaries & \textit{CHARACTER}(*): Specifies the boundaries of the geometry that must be forced to be conforming after the p-adaptation process.  	  &
					-- \\ \hline

order across faces &
			\textit{CHARACTER}: Mathematical expression to specify the maximum polynomial order jump across faces. Currently, only $N*2/3$ and $N-1$ are supported. &
					$N-1$ \\ \hline

mode &
			\textit{CHARACTER}: p-Adaptation mode. Can be \textit{static}, \textit{time} or \textit{iteration}. Static p-adaptation is performed once at the beginning of a simulation for steady or unsteady simulations. Unsteady adaptation can be performed by \textit{time} or by \textit{iteration}. &
					\textit{static} \\ \hline

interval &
			\textit{INTEGER/REAL}: In dynamic p-adaptation cases, this keyword specifies the iteration (integer) or time (real) interval for p-adaptation. &
					\textit{huge number} \\ \hline

restart files &
			\textit{LOGICAL}: If .TRUE., the program writes restart files before and after the p-adaptation. &
					.FALSE. \\ \hline

adjust nz &
    \textit{LOGICAL}: If .TRUE., the order across faces is adjusted in the directions xi, eta, and zeta of the face (being zeta the normal direction). If .FALSE., the     order is only adjusted in the xi and eta directions. The adjustment currently consists (hard-coded) in limiting the jumps in the polynomial order. &
            .FALSE. \\ \hline
\end{longtable}

\section{Truncation Error p-Adaptation}
This algorithm can perform a p-adaptation to decrease the truncation error below a threshold.
\begin{lstlisting}
#define p-adaptation
   adaptation type       = TE
   Truncation error type = isolated
   truncation error      = 1.d-2
   Nmax                  = [10,10,10]
   Nmin                  = [2 ,2 ,2 ]
   Conforming boundaries = [InnerCylinder,sphere]
   order across faces    = N*2/3
   increasing            = .FALSE.
   write error files     = .FALSE.
   adjust nz             = .FALSE.
   mode                  = time
   interval              = 1.d0
   restart files         = .TRUE.
   max N decrease        = 1
   padapted mg sweeps pre      = 10
   padapted mg sweeps post     = 12
   padapted mg sweeps coarsest = 20
#end
\end{lstlisting}

\begin{longtable}{|p{4cm}|p{10cm}|p{2.2cm}|}
\caption{Keywords for the TE p-adaptation algorithm.} \label{tab:TEpAdaptationKey} \\
\hline
\multicolumn{1}{|c|}{\textbf{Keyword}} & \multicolumn{1}{c|}{\textbf{Description}} & \multicolumn{1}{c|}{\textbf{Default value}} \\ \hline
\endfirsthead

\caption{Keywords for the TE p-adaptation algorithm - continued.} \\
\hline
\multicolumn{1}{|c|}{\textbf{Keyword}} & \multicolumn{1}{c|}{\textbf{Description}} & \multicolumn{1}{c|}{\textbf{Default value}} \\ \hline
\endhead

truncation error type & \textit{CHARACTER}: Can be either "isolated" or "non-isolated". & isolated \\ \hline

truncation error & \textit{REAL}: Target truncation error for the p-adaptation algorithm. & \textbf{Mandatory keyword} \\ \hline

coarse truncation error & \textit{REAL}: Truncation error used for coarsening. & same as truncation error \\ \hline

increasing & \textit{LOGICAL}: If .TRUE. the multi-stage FMG adaptation algorithm is used. &
					.FALSE. \\ \hline

write error files &
			\textit{LOGICAL}: If .TRUE., the program writes a file per element containing the directional tau-estimations. The files are stored in the folder \textit{./TauEstimation/}. When the simulation has several adaptation stages, the new information is just appended. &
			 		.FALSE. \\ \hline

max N decrease &
			\textit{INTEGER}: Maximum decrease in the polynomial order in every p-adaptation procedure. &
					$N-N_{\textit{min}}$ \\ \hline

post smoothing residual &
			\textit{REAL}: Specifies the maximum allowable deviation of $\partial_t q$ after the p-adaptation procedure. &
					-- \\ \hline

post smoothing method &
			\textit{CHARACTER}: Either RK3 or FAS. &
					RK3, if the last keyword is activated \\ \hline

estimation files &
			\textit{CHARACTER}: Name of the folder that contains the error estimations obtained with the multi tau-estimation (section \ref{sec:MultiTau}). &
					-- \\ \hline

estimation files number &
			\textit{INTEGER(2)}: First and last estimation stages to be used for p-adaptation. &
					Mandatory if last keyword is used. \\ \hline

padapted $\ll \textit{keyword} \gg$ &
			\textit{MULTIPLE}: Specifies control file keywords that should be replaced after the adaptation procedure. Currently, only 'mg sweeps         ', 'mg sweeps pre', 'mg sweeps post', and 'mg sweeps coarsest' are supported. &
					-- \\ \hline

\end{longtable}

\section{Reinforcement Learning p-Adaptation}
This algorithm can perform a p-adaptation based on a trained RL agent.

\begin{lstlisting}
#define p-adaptation
   adaptation type       = RL
   agent file            = policy_padaptation/p_adaptation_policy
   tolerance             = 1d-2
   Nmax                  = [6, 6, 2]
   Nmin                  = [2, 2, 2]
   Conforming boundaries = [cylinder]
   restart files         = .FALSE.
   mode                  = iteration
   interval              = 10
   threshold             = 2.0
#end
\end{lstlisting}

\begin{longtable}{|p{4cm}|p{10cm}|p{2.2cm}|}
\caption{Keywords for the RL p-adaptation algorithm.} \label{tab:RLpAdaptationKey} \\
\hline
\multicolumn{1}{|c|}{\textbf{Keyword}} & \multicolumn{1}{c|}{\textbf{Description}} & \multicolumn{1}{c|}{\textbf{Default value}} \\ \hline
\endfirsthead

\caption{Keywords for the RL p-adaptation algorithm - continued.} \\
\hline
\multicolumn{1}{|c|}{\textbf{Keyword}} & \multicolumn{1}{c|}{\textbf{Description}} & \multicolumn{1}{c|}{\textbf{Default value}} \\ \hline
\endhead

agent file & \textit{CHARACTER}: Relative path to the binary file that provide the policy of the RL agent. & \textbf{Mandatory keyword} \\ \hline

tolerance & \textit{REAL}: Tolerance for the RL agent. The smaller the tolerance, the wider the adapted region. & 0.01 \\ \hline

threshold & \textit{REAL}: The mesh will be adapted only if the percentage of the elements that require adaptation (in relation to the total number of elements) is above the threshold. & 0.0 \\ \hline


\end{longtable}


\section{Multiple truncation error estimations} \label{sec:MultiTau}
When using Truncation Error based adaptation, a static p-adaptation procedure can be driven by a set of error estimations, which have to be performed beforehand in a simulation with the following block:

\begin{lstlisting}
#define multi tau-estimation
   truncation error type = isolated
   interval              = 10
   folder                = MultiTau
#end
\end{lstlisting}

\section{Overenriching p-Adaptation}
It is possible to define one or more boxes where the user can select the polynomial order and bypass the adaptation process. This functionality can be used only if a p-adaptation algorithm is being used at the same time.

\begin{lstlisting}
#define overenriching box 1
   mode   = set
   order  = 4
   x span = [-1, 1]
   y span = [-1, 1]
   z span = [-1, 1]
#end
\end{lstlisting}

\begin{longtable}{|p{4cm}|p{10cm}|p{2.2cm}|}
\caption{Keywords for the overenriching p-adaptation algorithm.} \label{tab:RLpAdaptationKey} \\
\hline
\multicolumn{1}{|c|}{\textbf{Keyword}} & \multicolumn{1}{c|}{\textbf{Description}} & \multicolumn{1}{c|}{\textbf{Default value}} \\ \hline
\endfirsthead

\caption{Keywords for the overenriching p-adaptation algorithm - continued.} \\
\hline
\multicolumn{1}{|c|}{\textbf{Keyword}} & \multicolumn{1}{c|}{\textbf{Description}} & \multicolumn{1}{c|}{\textbf{Default value}} \\ \hline
\endhead

mode & \textit{CHARACTER}: Overenriching mode. Can be \textit{set} or \textit{increase}. The \textit{set} mode will fix the polynomial specified by the \textit{order} keyword inside the box, but it is constrained by the bounds specified by the \textit{Nmax} and \textit{Nmin} keywords. The \textit{increase} mode will increase the polynomial order selected by the algorithm (each time a p-adaptation is performed) the amount specified by the \textit{order} keyword. In this second mode, \textit{order} may be a possitive or negative value. & \textit{increase} \\ \hline

order & \textit{INTEGER}: Polynomial order to be used for the overenriching process. Its meaning depends on the selected \textit{mode}. & 1 \\ \hline

x span & \textit{REAL(2)}: The boundaries of the box in the x direction. & \textbf{Mandatory keyword} \\ \hline

y span & \textit{REAL(2)}: The boundaries of the box in the y direction. & \textbf{Mandatory keyword} \\ \hline

z span & \textit{REAL(2)}: The boundaries of the box in the z direction. & \textbf{Mandatory keyword} \\ \hline


\end{longtable}

%%%%%%%%%%%%%%%%%%%%%%%%%%%%%%%%%%%%%%%%%%%%%%%%%%%%%%%%%%%%%%%%%%%%%%%%

\chapter{Immersed boundary method}

The immersed boundary is activated during the simulation if the following lines are specified in the control file:
\begin{lstlisting}
#define IBM
   name                           = myIBM
   active                         = .true.
   penalization                   = 1.0d-6
   semi implicit                  = .false.
   number of objects              = 5 
   number of interpolation points = 15
   band region                    = .true.
   band region coefficient        = 1.3
   compute distance               = .true.
   clip axis                      =  1
   aab                            = .false.
   describe                       = .true.
   plot obb                       = .false.
   plot kdtree                    = .false.
   plot mask                      = .true.
   plot band points               = .false.
#end
\end{lstlisting}

A folder called 'IBM' must be created.

\begin{longtable}{|p{4cm}|p{10cm}|p{2.2cm}|}
\caption{Keywords for the immersed boundary method.} \label{tab:IBMtab} \\
\hline
\multicolumn{1}{|c|}{\textbf{Keyword}} & \multicolumn{1}{c|}{\textbf{Description}} & \multicolumn{1}{c|}{\textbf{Default value}} \\ \hline
\endfirsthead

\caption{Keywords for the immersed boundary method - continued.} \\
\hline
\multicolumn{1}{|c|}{\textbf{Keyword}} & \multicolumn{1}{c|}{\textbf{Description}} & \multicolumn{1}{c|}{\textbf{Default value}} \\ \hline
\endhead

name & \textit{CHARACTER}: name assigned to immersed boundary method job &  \\ \hline

active & \textit{LOGICAL}: When .TRUE. the immersed boundary method is active. & .FALSE. \\ \hline

penalization & \textit{REAL}: Specifies the value of the penalization term, \textit{i.e.} \eta. & \Delta t \\ \hline

semi implicit & \textit{LOGICAL}: The source term is treated in a semi-implicit manner. & .FALSE. \\ \hline


number of objects & \textit{INTEGER}: Specifies the maximum number of objects inside a leaf of the KD-tree. & 5 \\ \hline

number of interpolation points & \textit{INTEGER}: Number of points used for the interpolation of the variables' values on the surface. It's needed for the computation of the forces.  & 15 \\ \hline

band region & \textit{LOGICAL}: If it's true, the band region is computed, otherwise it is not. & .FALSE. \\ \hline

band region coefficient  & \textit{INTEGER}: A region $n$-times the oriented bounding box is created (where $n$ is the band region coefficient): all the points inside this region belong to the band region. The band region's points are used for the interpolation. It must be greater then 1 and '\textit{band region}' must be .true. . & 1.5 \\ \hline

compute distance & \textit{LOGICAL}: If it's true, the distance between the points in the band region and the stl file is computed, otherwise it is not. If the distance is not required, turn it off since it is an expansive operation. & .FALSE. \\ \hline

clip axis  & \textit{INTEGER}: It's the axis along which the stl is cut. It is only needed if the forces are computed so that the integration of the variables is performed only on the portion of the stl surface lying inside the mesh. 1 corresponds to \textit{x-axis}, 2 with \textit{y-axis} and 3 with \textit{z-axis}&  0 \\ \hline

aab  & \textit{LOGICAL}: The Axis Aligned Box is computed instead of the Oriented Bounding box. It is recommended when '\textit{clip axis}' $\ne$ 0. & .FALSE. \\ \hline

describe & \textit{LOGICAL}: The immersed boundary parameters are printed on the screen. & .FALSE. \\ \hline

plot obb & \textit{LOGICAL}: The oriented-bounding box is plotted. & .FALSE. \\ \hline

plot kdtree & \textit{LOGICAL}: The kd-tree is plotted. & .FALSE. \\ \hline

plot mask & \textit{LOGICAL}: The degrees of freedom belonging to the mask are plotted. & .FALSE. \\ \hline

plot band points & \textit{LOGICAL}: The band region's points are plotted. & .FALSE. \\ \hline

\end{longtable}


\subsection{STL file}

Immersed boundary requires, along with the mesh, a STL file. It must be put in the MESH folder with the mesh. The STL file name must be in lowercase character. In some programs, like AutoCAD, a STL file has always positive coordinates: the mesh should be built according to this consideration.\\
In the case of 2D simulations, the STL can be automatically cut by horses3D through the addition of the line \textit{clip axis} (described in the previous section) so that only the STL portions inside the mesh are considered.

\begin{longtable}{|p{4cm}|p{10cm}|p{2.2cm}|}
\caption{Keywords for stl files.} \label{tab:IBMstl} \\
\hline
\multicolumn{1}{|c|}{\textbf{Keyword}} & \multicolumn{1}{c|}{\textbf{Description}} & \multicolumn{1}{c|}{\textbf{Default value}} \\ \hline
\endfirsthead

\caption{Keywords for stl files - continued.} \\
\hline
\multicolumn{1}{|c|}{\textbf{Keyword}} & \multicolumn{1}{c|}{\textbf{Description}} & \multicolumn{1}{c|}{\textbf{Default value}} \\ \hline
\endhead

number of stl = & \textit{INTEGER}: Number of stl files. & 1 \\ \hline

stl file name = & \textit{CHARACTER}: The name of the STL file, without extension; it has to be inside the folder "MESH" . & \textbf{Mandatory keyword} \\ \hline

stl file nameN = & \textit{CHARACTER}: The name of the $\mathrm{N}^{th}$-STL file (where N starts from 2), without extension; for the first STL just use "stl file name". It has to be inside the folder "MESH" . & none \\ \hline
 

\end{longtable}


\subsection{Computing forces}

In order to compute the forces on a body, the monitor should be defined as usual but the "Marker=" has to be equal to the name of the stl file on which the user wants to compute the forces. Given a STL file called "stlname", the monitor should be:
\begin{lstlisting}
#define surface monitor 1
   marker = stlname
   .
   .
   .
#end
\end{lstlisting}
The result of this operation is a \textit{.tec} file inside the RESULTS folder. This file contains a scalar or a vector data projected on the body surface. 

\subsection{Moving bodies}

If one or more of the stl files move, then the following lines must be added:
\begin{lstlisting}
#define stl motion 1
   stl name         = mySTL
   type             = rotation 
   angular velocity = 134.5d0
   motion axis      = 2
#end
\end{lstlisting}

\begin{longtable}{|p{4cm}|p{10cm}|p{2.2cm}|}
\caption{Keywords for a moving stl.} \label{tab:movingstl} \\
\hline
\multicolumn{1}{|c|}{\textbf{Keyword}} & \multicolumn{1}{c|}{\textbf{Description}} & \multicolumn{1}{c|}{\textbf{Default value}} \\ \hline
\endfirsthead

\caption{Keywords for a moving stl - continued.} \\
\hline
\multicolumn{1}{|c|}{\textbf{Keyword}} & \multicolumn{1}{c|}{\textbf{Description}} & \multicolumn{1}{c|}{\textbf{Default value}} \\ \hline
\endhead

stl name & \textit{CHARACTER}: name of the moving stl; it has to be equal to the name of one of the stl files. & \textbf{mandatory keyword} \\ \hline

type & \textit{CHARACTER}: Type of motion, it can be ROTATION or LINEAR. & \textbf{Mandatory keyword} \\ \hline

angular velocity & \textit{REAL}: Specifies the angular velocity. It must be in [Rad]/[s] & \textbf{Mandatory keyword for rotation type} \\ \hline

velocity & \textit{REAL}: Specifies the translation velocity. It must be in [m]/[s] & \textbf{Mandatory keyword for linear type} \\ \hline

motion axis & \textit{REAL}: Specifies the axis along which the rotation/translation occurs. & \textbf{Mandatory keyword} \\ \hline

\end{longtable}

%%%%%%%%%%%%%%%%%%%%%%%%%%%%%%%%%%%%%%%%%%%%%%%%%%%%%%%%%%%%%%%%%%%%%%%%


\chapter{Monitors}~\label{sect:monitors}

The monitors are specified individually as blocks in the control file.
The only general keyword that can be specified is explained in Table \ref{tab:monitorsKey}.



\begin{table}[h]
\caption{Keywords for monitors.} \label{tab:monitorsKey}

\begin{tabular}{|p{4cm}|p{10cm}|p{2.2cm}|}
\hline
\multicolumn{1}{|c|}{\textbf{Keyword}} & \multicolumn{1}{c|}{\textbf{Description}} & \multicolumn{1}{c|}{\textbf{Default value}} \\ \hline

monitors flush interval 	&
			\textit{INTEGER}: Iteration interval to flush the monitor information to the monitor files. &
							100 \\ \hline

\end{tabular}
\end{table}

\section{Residual Monitors}

\section{Statistics Monitor}
\begin{lstlisting}
#define statistics
   initial time      = 1.d0
   initial iteration = 10
   sampling interval = 10
   dump interval     = 20
   @start
#end
\end{lstlisting}

By default, the statistic monitor will average following variables:

\begin{multicols}{3}
\begin{itemize}
\item u
\item v
\item w
\item uu
\item vv
\item ww
\item uv
\item uw
\item vw
\end{itemize}
\end{multicols}

A keyword preceded by @ is used in real-time to signalize the solver what it must do with the statistics computation:

\begin{multicols}{3}
\begin{itemize}
\item @start
\item @pause
\item @stop
\item @reset
\item @dump
\end{itemize}
\end{multicols}

After reading the keyword, the solver performs the desired action and marks it with a star, e.g. @start*.

\textbf{ATTENTION:} Real-time keywords may not work in parallel MPI computations. I depends on how the system is configured.

%%%%%%%%%%%%%%%%%%%%%%%%%%%%%%%%%%%%%%%%%%%%%%%%%%%%%%%%%%%%%%%%%%%%%%

\section{Probes}

\begin{lstlisting}
#define probe 1
   name     = SomeName
   variable = SomeVariable
   position = [0.d0, 0.d0, 0.d0]
#end
\end{lstlisting}

\begin{longtable}{|p{4cm}|p{10cm}|p{2.2cm}|}
\caption{Keywords for probes.} \label{tab:ProbesKey} \\
\hline
\multicolumn{1}{|c|}{\textbf{Keyword}} & \multicolumn{1}{c|}{\textbf{Description}} & \multicolumn{1}{c|}{\textbf{Default value}} \\ \hline
\endfirsthead

\caption{Keywords for the p-adaptation algorithms - continued.} \\
\hline
\multicolumn{1}{|c|}{\textbf{Keyword}} & \multicolumn{1}{c|}{\textbf{Description}} & \multicolumn{1}{c|}{\textbf{Default value}} \\ \hline
\endhead

name 	&
			\textit{CHARACTER}: Name of the monitor. &
							\textbf{Mandatory Keyword} \\ \hline

variable 	&
			\textit{CHARACTER}: Variable to be monitored. Implemented options are:\
\begin{multicols}{3}
\begin{itemize}
\item pressure
\item velocity
\item u
\item v
\item w
\item mach
\item k
\end{itemize}
\end{multicols}
			 &
							\textbf{Mandatory Keyword} \\ \hline

position 	&
			\textit{REAL(3)}: Coordinates of the point to be monitored. &
							\textbf{Mandatory Keyword} \\ \hline
\end{longtable}

%%%%%%%%%%%%%%%%%%%%%%%%%%%%%%%%%%%%%%%%%%%%%%%%%%%%%%%%%%%%%%%%%%%%%%%%

\section{Surface Monitors}


\begin{lstlisting}
#define surface monitor 1
   name              = SomeName
   marker            = NameOfBoundary
   variable          = SomeVariable
   reference surface = 1.d0
   direction         = [1.d0, 0.d0, 0.d0]
#end
\end{lstlisting}

\begin{longtable}{|p{4cm}|p{10cm}|p{2.2cm}|}
\caption{Keywords for probes.} \label{tab:SurfaceMonitorKey} \\
\hline
\multicolumn{1}{|c|}{\textbf{Keyword}} & \multicolumn{1}{c|}{\textbf{Description}} & \multicolumn{1}{c|}{\textbf{Default value}} \\ \hline
\endfirsthead

\caption{Keywords for the p-adaptation algorithms - continued.} \\
\hline
\multicolumn{1}{|c|}{\textbf{Keyword}} & \multicolumn{1}{c|}{\textbf{Description}} & \multicolumn{1}{c|}{\textbf{Default value}} \\ \hline
\endhead

name 	&
			\textit{CHARACTER}: Name of the monitor. &
							\textbf{Mandatory Keyword} \\ \hline

marker &
			\textit{CHARACTER}: Name of the boundary where a variable will be monitored. &
							\textbf{Mandatory Keyword} \\ \hline

variable 	&
			\textit{CHARACTER}: Variable to be monitored. Implemented options are:\
\begin{multicols}{2}
\begin{itemize}
\item mass-flow
\item flow
\item pressure-force
\item viscous-force
\item force
\item lift
\item drag
\item pressure-average
\end{itemize}
\end{multicols}
			 &
							\textbf{Mandatory Keyword} \\ \hline

reference surface 	&
			\textit{REAL}: Reference surface [area] for the monitor. Needed for "lift" and "drag" computations. &
							-- \\ \hline

direction 	&
			\textit{REAL(3)}: Direction in which the force is going to be measured. Needed for "pressure-force", "viscous-force" and "force". Can be specified for "lift" (default [0.d0,1.d0,0.d0]) and "drag" (default [1.d0,0.d0,0.d0])   &
							-- \\ \hline
\end{longtable}


\section{Volume Monitors}
Volume monitors compute the average of a quantity in the whole domain. They can be scalars(s) or vectors(v).

\begin{lstlisting}
#define volume monitor 1
   name     = SomeName
   variable = SomeVariable
#end
\end{lstlisting}

\begin{longtable}{|p{4cm}|p{10cm}|p{2.2cm}|}
\caption{Keywords for volume monitors.} \label{tab:VolMonitorsKey} \\
\hline
\multicolumn{1}{|c|}{\textbf{Keyword}} & \multicolumn{1}{c|}{\textbf{Description}} & \multicolumn{1}{c|}{\textbf{Default value}} \\ \hline
\endfirsthead

\caption{Keywords for volume monitors - continued.} \\
\hline
\multicolumn{1}{|c|}{\textbf{Keyword}} & \multicolumn{1}{c|}{\textbf{Description}} & \multicolumn{1}{c|}{\textbf{Default value}} \\ \hline
\endhead

name 	&
			\textit{CHARACTER}: Name of the monitor. &
							\textbf{Mandatory Keyword} \\ \hline

variable 	&
			\textit{CHARACTER}: Variable to be monitored. The variable can be scalar (s) or vectorial (v). Implemented options are:\
\begin{multicols}{2}
\begin{itemize}
\item[\textbf{(s)}] kinetic energy
\item[\textbf{(s)}] kinetic energy rate
\item[\textbf{(s)}] enstrophy
\item[\textbf{(s)}] entropy
\item[\textbf{(s)}] entropy rate
\item[\textbf{(s)}] mean velocity
\item[\textbf{(v)}] velocity
\item[\textbf{(v)}] momentum
\item[\textbf{(v)}] source
\end{itemize}
\end{multicols}

			 &
							\textbf{Mandatory Keyword} \\ \hline


\end{longtable}

\section{Load Balancing Monitors}
Load balancing monitors compute the DOF in each partition of the mesh to check the unbalance. 

\begin{lstlisting}
#define load balancing monitor 1
   name     = SomeName
   variable = SomeVariable
#end
\end{lstlisting}

\begin{longtable}{|p{4cm}|p{10cm}|p{2.2cm}|}
\caption{Keywords for load balancing monitors.} \label{tab:VolMonitorsKey} \\
\hline
\multicolumn{1}{|c|}{\textbf{Keyword}} & \multicolumn{1}{c|}{\textbf{Description}} & \multicolumn{1}{c|}{\textbf{Default value}} \\ \hline
\endfirsthead

\caption{Keywords for load balancing monitors - continued.} \\
\hline
\multicolumn{1}{|c|}{\textbf{Keyword}} & \multicolumn{1}{c|}{\textbf{Description}} & \multicolumn{1}{c|}{\textbf{Default value}} \\ \hline
\endhead

name 	&
			\textit{CHARACTER}: Name of the monitor. &
							\textbf{Mandatory Keyword} \\ \hline

variable 	&
			\textit{CHARACTER}: Variable to be monitored. Implemented options are:\
\begin{multicols}{2}
\begin{itemize}
\item max dof per partition
\item min dof per partition
\item avg dof per partition
\item absolute dof unbalancing
\item relative dof unbalancing
\end{itemize}
\end{multicols}

			 &
							\textbf{Mandatory Keyword} \\ \hline


\end{longtable}

The variable \emph{absolute dof unbalancing} is computed as follows:
\begin{equation}
    absolute\, dof\, unbalancing = max\,dof\,per\,partition - min\,dof\,per\,partition
\end{equation}
The variable \emph{relative dof unbalancing} is computed as follows:
\begin{equation}
    relative\, dof\, unbalancing = 100 * absolute\, dof\, unbalancing / avg\,dof\,per\,partition
\end{equation}

\chapter{Advanced User Setup}

Advanced users can have additional control over a simulation without having to modify the source code and recompile the code. To do that, the user can provide a set of routines that are called in different stages of the simulation via the Problem file (\textit{ProblemFile.f90}). A description of the routines of the Problem File can be found in section \ref{sec:ProblemFile}.

\section{Routines of the Problem File: \textit{ProblemFile.f90}} \label{sec:ProblemFile}

\begin{itemize}
\item UserDefinedStartup: Called before any other routines

\item UserDefinedFinalSetup: Called after the mesh is read in to allow mesh related initializations or memory allocations.

\item UserDefinedInitialCondition: called to set the initial condition for the flow. By default it sets an uniform initial condition, but the user can change it.

\item UserDefinedState1, UserDefinedNeumann: Used to define an user-defined boundary condition.

\item UserDefinedPeriodicOperation: Called before every time-step to allow periodic operations to be performed.

\item UserDefinedSourceTermNS: Called to apply source terms to the equation.

\item UserDefinedFinalize: Called after the solution computed to allow, for example error tests to be performed.

\item UserDefinedTermination: Called at the the end of the main driver after everything else is done.
\end{itemize}

\section{Compiling the Problem File}

The Problem File file must be compiled using a specific Makefile that links it with the libraries of the code. If you are using the \textit{horses/dev} environment module, you can get templates of the \textit{Problemfile.f90} and \textit{Makefile} with the following commands:

\begin{lstlisting}[language=bash]
	$ horses-get-makefile
	$ horses-get-problemfile
\end{lstlisting}

Otherwise, search the test cases for examples.\\

To run a simulation using user-defined operations, create a folder called SETUP on the path were the simulation is going to be run. Then, store the modified \textit{ProblemFile.f90} and the \textit{Makefile} in SETUP, and compile using:

\begin{lstlisting}[language=bash]
	$ make <<Options>>
\end{lstlisting}
where again the options are (bold are default):
\begin{itemize}
\item MODE=DEBUG/HPC/\textbf{RELEASE}
\item COMPILER=ifort/\textbf{gfortran}
\item COMM=PARALLEL/\textbf{SEQUENTIAL}
\item ENABLE\_THREADS=NO/\textbf{YES}
\end{itemize}

%%%%%%%%%%%%%%%%%%%%%%%%%%%%%%%%%%%%%%%%%%%%%%%%%%%%%%%%%%%%%%%%%%%%%%

\chapter{Postprocessing}

For postprocessing the Simulation Results

\section{Visualization with Tecplot Format: \textit{horses2plt}}

HORSES3D provides a script for converting the native binary solution files (*.hsol) into tecplot ASCII format (*.tec), which can be visualized in Pareview or Tecplot. It can also export the solution to the more recent VTKHDF format; however, note that this feature does \textbf{not} export boundary information or mesh files. Usage:

\begin{lstlisting}[language=bash]
	$ horses2plt SolutionFile.hsol MeshFile.hmesh <<Options>>
\end{lstlisting}

The options comprise following flags:

\begin{longtable}{|p{4cm}|p{10cm}|p{2.2cm}|}
\caption{Flags for \textit{horses2plt}.} \label{tab:Postprocessing} \\
\hline
\multicolumn{1}{|c|}{\textbf{Flag}} & \multicolumn{1}{c|}{\textbf{Description}} & \multicolumn{1}{c|}{\textbf{Default value}} \\ \hline
\endfirsthead

\caption{Additional flags for postprocessing with \textit{horses2plt}} \\
\hline
\multicolumn{1}{|c|}{\textbf{Flag}} & \multicolumn{1}{c|}{\textbf{Description}} & \multicolumn{1}{c|}{\textbf{Default value}} \\ \hline
\endhead

-{}-output-order= 	&
			\textit{INTEGER}: Output order nodes. The solution is interpolated into the desired number of points. &
							Not Present \\ \hline

-{}-output-basis= 	&
			\textit{CHARACTER}: Either \textit{Homogeneous} (for equispaced nodes, or \textit{Gauss}.  &
							\textit{Gauss}* \\ \hline

-{}-output-mode= 	&
			\textit{CHARACTER}: Either \textit{multizone} or \textit{FE}. The option \textit{multizone} generates a Tecplot zone for each element. The option \textit{FE} generates only one Tecplot zone for the fluid and one for each boundary (if \textit{-{}-boundary-file} is defined).
			Each subcell is mapped as a linear finite element. This format is faster to read by Paraview and Tecplot.  &
							\textit{multizone} \\ \hline

-{}-output-variables= 	&
			\textit{CHARACTER}: Output variables separated by commas.A complete description can be found in Section \ref{PostProc:hsol}. &
							Q \\ \hline

-{}-dimensionless 	&
			Specifies that the output quantities must be dimensionless &
							Not Present  \\ \hline

-{}-partition-file= 	&
			\textit{CHARACTER}: Specifies the path to the partition file (*.pmesh) to export the MPI ranks of the simulation. &
							Not Present  \\ \hline

-{}-boundary-file= 	&
			\textit{CHARACTER}: Specifies the path to the boundary mesh file (*.bmesh) to export the surfaces as additional zones of the Tecplot file. &
							Not Present  \\ \hline

-{}-output-type= 	&
            \textit{CHARACTER}: Specifies the type of output file: \textit{tecplot} or \textit{vtkhdf}. &
                            \textit{tecplot}  \\ \hline
\multicolumn{3}{p{16.4cm}}{*  \textit{Homogeneous} when \textit{-{}-output-order} is specified} \\

\end{longtable}

Additionally, depending on the type of solution file, the user can specify additional options.

\subsection{Solution Files (*.hsol)} \label{PostProc:hsol}

For standard solution files, the user can specify which variables they want to be exported to the Tecplot file with the flag \textit{-{}-output-variables=}.
The options are:

\begin{multicols}{5}
\begin{itemize}
\item $Q$ (default)
\item $rho$
\item $u$
\item $v$
\item $w$
\item $p$
\item $T$
\item $Mach$
\item $s$
\item $Vabs$
\item $V$
\item $Ht$
\item $rhou$
\item $rhov$
\item $rhow$
\item $rhoe$
\item $c$
\item $Nxi$
\item $Neta$
\item $Nzeta$
\item $Nav$
\item $N$
\item $Ax\_Xi$
\item $Ax\_Eta$
\item $Ax\_Zeta$
\item $ThreeAxes$
\item $Axes$
\item $mpi\_rank$
\item $eID$
\item $gradV$
\item $u\_x$
\item $v\_x$
\item $w\_x$
\item $u\_y$
\item $v\_y$
\item $w\_y$
\item $u\_z$
\item $v\_z$
\item $w\_z$
\item $c\_x$
\item $c\_y$
\item $c\_z$
\item $omega$
\item $omega\_x$
\item $omega\_y$
\item $omega\_z$
\item $omega\_abs$
\item $Qcrit$
\end{itemize}
\end{multicols}

\subsection{Statistics Files (*.stats.hsol)}
Statistics files can generate the standard variables as well as the following variables (being $S_{ij}$ the components of the Reynolds Stress tensor):

\begin{multicols}{3}
\begin{itemize}
\item $umean$
\item $vmean$
\item $wmean$
\item $S_{xx}$
\item $S_{yy}$
\item $S_{zz}$
\item $S_{xy}$
\item $S_{xz}$
\item $S_{yz}$
\end{itemize}
\end{multicols}

\section{Extract geometry}
Under construction.

\section{Merge statistics tool}

Tool to merge several statistics files. The usage is the following:

\begin{lstlisting}[language=bash]
	$ horses.mergeStats *.hsol --initial-iteration=INTEGER --file-name=CHARACTER
\end{lstlisting}

Some remarks:
\begin{itemize}
\item Only usable with statistics files that are obtained with the "reset interval" keyword and/or with individual consecutive simulations.
\item Only constant time-stepping is supported.
\item If the hsol files have the gradients, the following flag must be used
\begin{lstlisting}[language=bash]
	$ --has-gradients
\end{lstlisting}
\item Dynamic p-adaptation is currently not supported.
\end{itemize}

\section{Mapping result to different mesh} \label{tab:meshInterpolation}
HORSES3D addons, \textit{horsesConverter}, has a capability to map result into different mesh file, with both have a consistent geometry. This is done by performing interpolation with the polynomial inside each element for each node point of the new mesh. The type of node quadrature will follow the quadrature defined in the .hmesh file with selected polynomial order in the control file. A control input file is required and must has name \textit{horsesConverter.convert}. The template of control input file will be generated by default when executing \textit{./horsesConverter} in a directory without the control file. Error message is given when at least one node point of the new mesh is not within any element of the old mesh. After completion, a new result file is generated and named \textit{Result\_interpolation.hsol}. The required keywords in the control file are described in table \ref{tab:mappingresult}. Command to execute:

\begin{lstlisting}[language=bash]
	$ ./horsesConverter
\end{lstlisting}

\begin{longtable}{|p{4cm}|p{10cm}|p{2.2cm}|}
\caption{Keywords for \textit{meshInterpolation}.} \label{tab:mappingresult} \\
\hline
\multicolumn{1}{|c|}{\textbf{Keyword}} & \multicolumn{1}{c|}{\textbf{Description}} & \multicolumn{1}{c|}{\textbf{Default value}} \\ \hline
\endfirsthead

Task 	        		& \textit{meshInterpolation} 						&  \\ \hline
Mesh Filename 1 		& Location of the origin mesh (*.hmesh) 	&  \\ \hline
Boundary Filename 1 	& Location of the origin boundary mesh (*.bmesh) 	&  \\ \hline
Result 1 			& Location of the solution file with origin mesh (*.hsol) 	&  \\ \hline
Mesh Filename 2 		& Location of the target mesh (*.hmesh) 	&  \\ \hline
Boundary Filename 2 	& Location of the target boundary mesh (*.bmesh) 	&  \\ \hline
Polynomial Order 	& Polynomial order of the target mesh 	& (1, 1, 1) \\ \hline
\end{longtable}

\section{Generate OpenFOAM mesh} \label{tab:horsesMesh2OF}
Another functionality of \textit{horsesConverter} addons is to convert the mesh files, (*.hmesh) and (*.bmesh), into OpenFOAM format, the \textit{polyMesh} folder. Each element is discretised into $n_x \times n_y \times n_z$ cells distributed as Gauss-Lobatto nodes. The number of division of each element, ($n_x$, $n_y$, and $n_z$), is required in the control file, see section \ref{tab:meshInterpolation}. After completion, a folder named \emph{foamFiles} is generated. OpenFOAM mesh files, i.e. \textit{points}, \textit{faces}, \textit{owner}, \textit{neighbour}, and \textit{boundary}, are located within \emph{foamFiles/constant/polyMesh}. The required keywords in the control file are described in table \ref{tab:horsesMesh2of}. Command to execute:

\begin{lstlisting}[language=bash]
	$ ./horsesConverter
\end{lstlisting}

\begin{longtable}{|p{4cm}|p{10cm}|p{2.2cm}|}
\caption{Keywords for \textit{horsesMesh2OF}.} \label{tab:horsesMesh2of} \\
\hline
\multicolumn{1}{|c|}{\textbf{Keyword}} & \multicolumn{1}{c|}{\textbf{Description}} & \multicolumn{1}{c|}{\textbf{Default value}} \\ \hline
\endfirsthead

Task 	        		& \textit{horsesMesh2OF} 						&  \\ \hline
Mesh Filename 1 		& Location of the origin mesh (*.hmesh) 	&  \\ \hline
Boundary Filename 1 	& Location of the origin boundary mesh (*.bmesh) 	&  \\ \hline
Polynomial Order 	& Number of division of each element ($n_x$, $n_y$, and $n_z$) 	& (1, 1, 1) \\ \hline
\end{longtable}

NOTE: Before running the mesh in the OpenFOAM environment, the type of boundaries inside the boundary file needs to be adjusted according to the actual type (\textit{patch}, \textit{wall}, and \textit{symmetry}).

\section{Generate HORSES3D solution file from OpenFOAM result}
HORSES3D provides a capability to convert OpenFOAM result into HORSES3D solution file (*.hsol). The mesh of the OpenFOAM result must be generated by converting HORSES3D mesh files, see section \ref{tab:horsesMesh2OF}. Beforehand, the OpenFOAM result must be converted into VTK format(*.vtk). This not only  allows the result to be in the single file but also converts cell data into point data. In the OpenFOAM environment, the command for this conversion:  
\begin{lstlisting}[language=c++]
	$ foamToVTK -fields "(U p T rho)" -ascii -latestTime
\end{lstlisting}
The necessary file (.vtk) required for the control file input is inside VTK folder, see section \ref{tab:meshInterpolation} for the control file template. The HORSES3D solution file is named \emph{Result\_OF.hsol}. The required keywords in the control file are described in table \ref{tab:OF2Horses}. Command to execute:

\begin{lstlisting}[language=bash]
	$ ./horsesConverter
\end{lstlisting}


\begin{longtable}{|p{5cm}|p{9cm}|p{2.2cm}|}
\caption{Keywords for \textit{OF2Horses}.} \label{tab:OF2Horses} \\
\hline
\multicolumn{1}{|c|}{\textbf{Keyword}} & \multicolumn{1}{c|}{\textbf{Description}} & \multicolumn{1}{c|}{\textbf{Default value}} \\ \hline
\endfirsthead

\caption{Keywords for \textit{OF2Horses} - continued.} \\
\hline
\multicolumn{1}{|c|}{\textbf{Keyword}} & \multicolumn{1}{c|}{\textbf{Description}} & \multicolumn{1}{c|}{\textbf{Default value}} \\ \hline
\endhead

Task 	        		& \textit{OF2Horses} 						&  \\ \hline
Mesh Filename 1 		& Location of the origin mesh (*.hmesh) 	&  \\ \hline
Boundary Filename 1 	& Location of the origin boundary mesh (*.bmesh) 	&  \\ \hline
Polynomial Order 	& Polynomial order of the solution file (.hsol)  & (1, 1, 1) \\ \hline
VTK file         	& Location of VTK file (.vtk)   &             \\ \hline
Reynolds Number    	& Reynolds Number/m of the solution -- $L_{ref}$=1.0m   &             \\ \hline
Mach Number         	& Mach Number of the solution      &             \\ \hline
Reference pressure (Pa)         	& Reference Pressure   & 101325             \\ \hline
Reference temperature (K)        & Reference Temperature   & 288.889             \\ \hline

\end{longtable}

%%%%%%%%%%%%%%%%%%%%%%%%%%%%%%%%%%%%%%%%%%%%%%%%%%%%%%%%%%%%%%%%%%%%%%

\chapter{Performance}
\section{Hybrid MPI and OMP}
Balancing the utilization of MPI and OMP in extensive simulations, particularly for large cases with a degree of freedom (NDOF) exceeding 5,000,000, alongside a substantial number of CPU cores, is advisable. Figure 15.1 illustrates the relative speed-up achieved with varying ratios of shared faces in partitioning and the total number of cores. The hybrid setup demonstrates peak performance at approximately $\left(\frac{nMPIFacesShared}{nMPIFaces}\right)_{max} \approx 0.4-0.6$, particularly when the total number of cores is high. Utilizing this value as a reference point is recommended. In smaller cases, maximizing the number of OMP is recommended. 
\begin{figure}[h]
\centering
        \begin{tikzpicture}
            \begin{axis}[
                xlabel={$\left(\frac{nMPIFacesShared}{nMPIFaces}\right)_{max}$},
                ylabel={Relative Speed-Up},
                label style={font=\small},
                xmin=0, xmax=1,
                ymin=0.6, ymax=1.7,
                axis lines=left,
                axis line style={-stealth, line width=2pt}, % Adjust arrow size and line width
                xtick={0, 0.2, 0.4, 0.6, 0.8, 1.0},
                ytick={0.5,1.0,1.5},
                ylabel style={at={(ticklabel cs:0.3)}, anchor=south west, inner sep=0pt, yshift=-10mm}, % Adjust the position of the y-axis label
                grid=both,
                ymajorgrids=true,
                xmajorgrids=true, % Display only vertical grid lines
                major grid style={line width=0.2pt, grey!50},
                width=0.48\textwidth,
                legend pos=south east
            ]
                % Plot the curve using the data points
        		\addplot[blue, thick, mark=*] coordinates {(0.029829545,1.05811138) (0.111114264,1.179805616) (0.216105632,0.957074025) (0.390092879,0.947117469) (0.652849741,1)};
        		\addlegendentry{nCores = 32}
        		
                \addplot[red, dashed, thick, mark=x] coordinates {(0.111114264,1.31696086) (0.216105632,1.462915601) (0.390092879,1.600746269) (0.652849741,1.383870968) (0.984375,1)};
                \addlegendentry{nCores = 64}                
                
                \addplot[black, dashdotted, thick, mark=o] coordinates {(0.216105632,1.237316725) (0.390092879,1.37962963) (0.652849741,1.281892728) (0.984375,1)};
                \addlegendentry{nCores = 128}  
                
                % Box Plot
                \draw[line width=1pt] (current axis.south west) rectangle (current axis.north east); % Box around the axis
            \end{axis}
        \end{tikzpicture}
        \label{fig: OMP_MPI_sharedfaces}
\caption{Relative speed-up in Hybrid MPI-OMP based on maximum ratio of faces in partitions. NS solver with $64\times64\times32$ elements and polynomial N=3. }
\end{figure}



\clearpage
\section*{Appendices}

\subsection*{Non-dimensional Navier Stokes equations}

To illustrate the roles of various terms in the governing equations, we present here the non-dimensionalized governing compressible Navier Stokes equations with particles. We define the non-dimensional variables as
\begin{equation}
x_i^*=x_i/L_o, \quad y_i^*=y_i/L_o, \quad t^*=tu_o/L_o, \quad \rho^*=\rho/\rho_o, \quad u_i^*=u_i/u_o	
\end{equation}
\begin{equation}
v_i^*=v_i/u_o, \quad p^*=p/\rho_o u_o^2, \quad e^*=e/u_o^2 \quad T^*=T/T_o, \quad T_p^* = T_p / T_o, 
\end{equation}
\begin{equation}
\quad \mu^* = \mu / \mu_o, \quad \kappa^* = \kappa / \kappa_o \quad g^*_i = g_i / g_o	
\end{equation}

where the subscript $o$ denotes a reference value. Under these scalings, the Navier Stokes equations become
\begin{equation}
\label{eq:NS_cont_nd}
\frac{\partial \rho^*}{\partial t^*} + \frac{\partial \rho^* u_j^*}{\partial x_j^*} = 0,	
\end{equation}
\begin{equation}
\label{eq:NS_mom_nd}
\frac{\partial \rho^* u_i^*}{\partial t^*} + \frac{\partial \rho^* u_i^* u_j^*}{\partial x_j^*} = - \frac{\partial p^*}{\partial x_i^*} + \frac{1}{Re}\frac{\partial \tau_{ij}^*}{\partial x_j^*} + \frac{1}{Fr^2} g_i^* - \beta\frac{\mu^* \phi_m}{N_p St}  \sum_{n=1}^{N_p} (u_i^*-v_{i,n}^*)\delta(\mathbf{x}^* - \mathbf{y}_n^* ),	
\end{equation}
\begin{equation}
\label{eq:NS_en_nd}
\begin{split}
\frac{\partial \rho^* e^*}{\partial t^*} + \frac{\partial u_j^* (\rho^* e^* + p^*)}{\partial x_j^*} = & \frac{1}{Re}\left[\frac{\partial \tau_{ij}^* u_j^*}{\partial x_i^*} + \frac{1}{(\gamma-1)Pr M_o^2}\frac{\partial}{\partial x_j^*}\left( k^* \frac{\partial T^*}{\partial x_j^*}\right)\right]+ \\
& + \beta \frac{\phi_m}{3N_p} \frac{Nu}{(\gamma-1)Pr M_o^2St}\sum_{n=1}^{N_p} (T_{p,n}^* - T^*) \delta(\mathbf{x}^* - \mathbf{y}_n^* ).
\end{split}
\end{equation}

Where $Re=\frac{\rho_o L_o u_o}{\mu_o}$ is the Reynolds number, $Fr=\frac{u_o}{\sqrt{g_o L_o}}$ is the Froude number, $\phi_m=\frac{m_p N_p}{\rho_o L_o^3}$, $St=\frac{\tau_p}{\tau_f}$ is the Stokes number which for Stokesian particles is $St=\frac{\rho_p D_p^2 u_o}{18 L_o \mu_o}$, the Prandtl number (which is assumed constant and equal to 0.72) $Pr=\frac{\mu_o c_p}{k_o}$, the Nusselt number $Nu=\frac{h D_p}{k_o}=2$ (for Stokesian particles) and  the Mach number $M_o=\frac{u_o}{\sqrt{\gamma R T_o}}$. 


\subsection*{Non-dimensional particle equations}

The non dimensional set of equations for the particles reads:

\begin{equation}
\label{eq:part_motion}
\frac{d y_i^*}{dt^*} = v_i^*, \quad \frac{d v_i^*}{dt^*} = \frac{\mu^*}{St}\left(u_i^* - v_i^*\right) + \frac{1}{Fr^2}g_i^*.
\end{equation}

\begin{equation}
\label{eq:part_energy}
\frac{d T_p^*}{dt^*} = \frac{\gamma}{3\Gamma St Pr} \left(I_o^* - Nu(T_p^*-T^*)\right),
\end{equation}

where $I_o^*=\frac{I_o D_p}{4k_oT_o}$ and $\Gamma=c_{v,p}/c_{v}$ is the ratio of the particle specific heat capacity to the fluid isochoric specific heat capacity. 






\bibliography{../LaTeX/9_backmatter/library}

\end{document}
